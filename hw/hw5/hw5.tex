
\documentclass[letterpaper, reqno,11pt]{article}
\usepackage[margin=1.0in]{geometry}
\usepackage{color,latexsym,amsmath,amssymb}
\usepackage{fancyhdr}
\usepackage{amsthm}
\usepackage[linesnumbered,lined,boxed,commentsnumbered,noend,noline]{algorithm2e}
\usepackage{dsfont}
\usepackage{graphicx}
\usepackage{hyperref}
\usepackage{bbm}
\usepackage{enumitem}
\usepackage[numbers]{natbib}
\usepackage{framed}
\usepackage{titling}
\usepackage{subcaption}
\usepackage[dvipsnames]{xcolor}
\usepackage{tikz}

\tikzset{invclip/.style={clip,insert path={{[reset cm]
  (-16383.99999pt,-16383.99999pt) rectangle (16383.99999pt,16383.99999pt)}}}}

\allowdisplaybreaks
\DontPrintSemicolon
\SetKw{Break}{break}

\newcommand{\RR}{\mathbb{R}}
\newcommand{\CC}{\mathbb{C}}
\newcommand{\ZZ}{\mathbb{Z}}
\newcommand{\QQ}{\mathbb{Q}}
\newcommand{\NN}{\mathbb{N}}
\newcommand{\FF}{\mathbb{F}}
\newcommand{\PP}{\mathop{{}\mathbb{P}}}
\newcommand{\EE}{\mathop{{}\mathbb{E}}}
\newcommand{\inci}{\mathds{1}}
\DeclareMathOperator{\conv}{conv}
\DeclareMathOperator{\charcone}{char.cone}
\DeclareMathOperator{\STAB}{STAB}
\DeclareMathOperator{\Down}{Down}
\DeclareMathOperator{\lca}{lca}
\DeclareMathOperator{\ex}{ex}
\DeclareMathOperator{\Span}{span}
\DeclareMathOperator{\T}{T}
\DeclareMathOperator{\rank}{rank}
\DeclareMathOperator{\diag}{diag}
\DeclareMathOperator{\comp}{c}
\DeclareMathOperator{\bernoulli}{Bernoulli}
\DeclareMathOperator{\Mod}{mod}
\DeclareMathOperator{\sign}{sign}
\DeclareMathOperator{\LowestOne}{\textsc{LowestOne}}
\DeclareMathOperator{\dist}{dist}
\DeclareMathOperator{\poly}{poly}
\DeclareMathOperator{\argmin}{arg\,min}
\DeclareMathOperator{\error}{error}
\DeclareMathOperator{\NS}{\mathit{NS}}
\SetKwFor{RepTimes}{repeat}{times}{end}
\newcommand\mycommfont[1]{\footnotesize\ttfamily\textcolor{blue}{#1}}
\SetCommentSty{mycommfont}
\begin{document}
\pagenumbering{arabic}
\title{Homework 5}
\author{Yuchong Pan}
\date{\today}
\newtheorem{theorem}{Theorem}
\newtheorem{lemma}[theorem]{Lemma}
\newtheorem{corollary}[theorem]{Corollary}
\newtheorem{fact}[theorem]{Fact}
\newtheorem{proposition}[theorem]{Proposition}
\newtheorem{claim}{Claim}
\newtheorem{exercise}{Exercise}
\theoremstyle{definition}
\newtheorem{definition}[theorem]{Definition}
\newtheorem{solution}{Solution}
%\maketitle
%

\begin{framed}
\noindent{\bf 6.842 Randomness and Computation} \hfill \thedate
\begin{center}
\Large{\thetitle}
\end{center}
%\noindent{\em Lecturer: Ronitt Rubinfield} \hfill {\em Scribe: \theauthor}
\noindent{\em Yuchong Pan} \hfill {\em MIT ID: 911346847}
\end{framed}

\begin{enumerate}
  \item \noindent\emph{Collaborators and sources:} none.
  
  \begin{proof}
    Let $f : \{ \pm 1 \}^n \to \{ \pm 1 \}$. Let $\varepsilon \in (0, 1/2)$. Then
    \begin{align*}
      \NS_\varepsilon(f) &= \PP_{\substack{x \in \{ \pm 1 \}^n \\ N_\varepsilon}} \left[f(x) \neq f\left(N_\varepsilon(x)\right)\right] \\
      &= \PP_{\substack{x \in \{ \pm 1 \}^n \\ N_\varepsilon}} \left[f(x) f\left(N_\varepsilon(x)\right) = -1\right] \\
      &= \EE_{\substack{x \in \{ \pm 1 \}^n \\ N_\varepsilon}}\left[\frac{1}{2} - \frac{1}{2}f(x) f\left(N_\varepsilon(x)\right)\right] \\
      &= \frac{1}{2} - \frac{1}{2} \EE_{\substack{x \in \{ \pm 1 \}^n \\ N_\varepsilon}}\left[f(x) f\left(N_\varepsilon(x)\right)\right] \\
      &= \frac{1}{2} - \frac{1}{2} \EE_{\substack{x \in \{ \pm 1 \}^n \\ N_\varepsilon}}\left[\left(\sum_{S \subset [n]} \hat{f}(S) \chi_S(x)\right) \left(\sum_{T \subset [n]} \hat{f}(T) \chi_T\left(N_\varepsilon(x)\right)\right)\right] \\
      &= \frac{1}{2} - \frac{1}{2} \sum_{S, T \subset [n]} \hat{f}(S) \hat{f}(T) \EE_{\substack{x \in \{ \pm 1 \}^n \\ N_\varepsilon}}\left[\chi_S(x) \chi_T\left(N_\varepsilon(x)\right)\right].
    \end{align*}
    For all $x \in \{ \pm 1 \}^n$ and $i \in [n]$, we denote by $x_i$ and $N_\varepsilon(x)_i$ the $i^\text{th}$ coordinates of $x$ and $N_\varepsilon(x)$, respectively. For all $S \subset [n]$,
    \begin{align}
      \EE_{\substack{x \in \{ \pm 1 \}^n \\ N_\varepsilon}}\left[\chi_S(x) \chi_S\left(N_\varepsilon(x)\right)\right] &= \EE_{\substack{x \in \{ \pm 1 \}^n \\ N_\varepsilon}} \left[\left(\prod_{i \in S} x_i\right)\left(\prod_{i \in S} N_\varepsilon(x)_i\right)\right] \nonumber \\
      &= \EE_{\substack{x \in \{ \pm 1 \}^n \\ N_\varepsilon}}\left[\prod_{i \in S} x_i N_\varepsilon(x)_i\right] \nonumber \\
      &= \prod_{i \in S} \EE_{\substack{x \in \{ \pm 1 \}^n \\ N_\varepsilon}} \left[x_i N_\varepsilon(x)_i\right] \label{eq:1-indep} \\
      &= (\varepsilon \cdot (-1) + (1 - \varepsilon) \cdot 1)^{|S|} \label{eq:1-symmetry} \\
      &= (1 - 2\varepsilon)^{|S|}. \nonumber
    \end{align}
    Note that \eqref{eq:1-indep} is due to the independence of each bit in $N_\varepsilon(x)$, and \eqref{eq:1-symmetry} is due to the fact that each bit of $x$ uniformly chosen from $\{ \pm 1 \}^n$ is uniform in $\{ \pm 1 \}$. For all $S, T \subset [n]$ with $S \neq T$,
    \begin{align}
      &\quad\, \EE_{\substack{x \in \{ \pm 1 \}^n \\ N_\varepsilon}}\left[\chi_S(x) \chi_T\left(N_\varepsilon(x)\right)\right] \nonumber \\
      &= \EE_{\substack{x \in \{ \pm 1 \}^n \\ N_\varepsilon}} \left[\left(\prod_{i \in S} x_i\right)\left(\prod_{i \in T} N_\varepsilon(x)_i\right)\right] \nonumber \\
      &= \EE_{\substack{x \in \{ \pm 1 \}^n \\ N_\varepsilon}} \left[\left(\prod_{i \in S \cap T} x_i N_\varepsilon(x)_i\right) \left(\prod_{i \in S \setminus T} x_i\right)\left(\prod_{i \in T \setminus S} N_\varepsilon(x)_i\right)\right] \nonumber \\
      &= \left(\prod_{i \in S \cap T} \EE_{\substack{x \in \{ \pm 1 \}^n \\ N_\varepsilon}} \left[x_i N_\varepsilon(x)_i\right]\right) \left(\prod_{i \in S \setminus T} \EE_{x \in \{ \pm 1 \}^n}\left[x_i\right]\right) \left(\prod_{i \in T \setminus S} \EE_{\substack{x \in \{ \pm 1 \}^n \\ N_\varepsilon}}\left[N_\varepsilon(x)_i\right]\right). \label{eq:1-indep-diff}
    \end{align}
    Note that \eqref{eq:1-indep-diff} is again due to the independence of each bit in $N_\varepsilon(x)$. For $S, T \subset [n]$ with $S \neq T$, either $S \setminus T \neq \emptyset$ or $T \setminus S \neq \emptyset$. Note that each bit of $x$ uniformly chosen from $\{ \pm 1 \}^n$ is uniform in $\{ \pm 1 \}$. Therefore, if $S \setminus T \neq \emptyset$,
    \begin{align*}
      \prod_{i \in S \setminus T} \EE_{x \in \{ \pm 1 \}^n}\left[x_i\right] = \left(\EE_{b \in \{ \pm 1 \}}[b]\right)^{|S \setminus T|} = 0^{|S \setminus T|} = 0.
    \end{align*}
    Moreover, if $T \setminus S \neq \emptyset$,
    $$ \prod_{i \in T \setminus S} \EE_{\substack{x \in \{ \pm 1 \}^n \\ N_\varepsilon}}\left[N_\varepsilon(x)_i\right] = \left(\frac{1}{2}(\varepsilon (-1) + (1 - \varepsilon) \cdot 1) + \frac{1}{2} (\varepsilon \cdot 1 + (1 - \varepsilon) (-1))\right)^{|T \setminus S|} = 0^{|T \setminus S|} = 0. $$
    Therefore, for all $S, T \subset [n]$ with $S \neq T$,
    $$ \EE_{\substack{x \in \{ \pm 1 \}^n \\ N_\varepsilon}}\left[\chi_S(x) \chi_T\left(N_\varepsilon(x)\right)\right] = 0. $$
    It follows that
    \begin{align*}
      \NS_\varepsilon(f) = \frac{1}{2} - \frac{1}{2} \sum_{S, T \subset [n]} \hat{f}(S) \hat{f}(T) \EE_{\substack{x \in \{ \pm 1 \}^n \\ N_\varepsilon}}\left[\chi_S(x) \chi_T\left(N_\varepsilon(x)\right)\right] = \frac{1}{2} - \frac{1}{2} \sum_{S \subset [n]} \hat{f}(S)^2 (1 - 2 \varepsilon)^{|S|}.
    \end{align*}
    This completes the proof.
  \end{proof}
\end{enumerate}

\end{document}
