
\documentclass[letterpaper, reqno,11pt]{article}
\usepackage[margin=1.0in]{geometry}
\usepackage{color,latexsym,amsmath,amssymb}
\usepackage{fancyhdr}
\usepackage{amsthm}
\usepackage[linesnumbered,lined,boxed,commentsnumbered,noend,noline]{algorithm2e}
\usepackage{dsfont}
\usepackage{graphicx}
\usepackage{hyperref}
\usepackage{bbm}
\usepackage{enumitem}
\usepackage[numbers]{natbib}
\usepackage{framed}
\usepackage{titling}
\usepackage{subcaption}
\usepackage[dvipsnames]{xcolor}
\usepackage{tikz}

\tikzset{invclip/.style={clip,insert path={{[reset cm]
  (-16383.99999pt,-16383.99999pt) rectangle (16383.99999pt,16383.99999pt)}}}}

\allowdisplaybreaks

\newcommand{\RR}{\mathbb{R}}
\newcommand{\CC}{\mathbb{C}}
\newcommand{\ZZ}{\mathbb{Z}}
\newcommand{\QQ}{\mathbb{Q}}
\newcommand{\NN}{\mathbb{N}}
\newcommand{\FF}{\mathbb{F}}
\newcommand{\PP}{\mathbb{P}}
\newcommand{\EE}{\mathbb{E}}
\newcommand{\inci}{\mathds{1}}
\DeclareMathOperator{\conv}{conv}
\DeclareMathOperator{\charcone}{char.cone}
\DeclareMathOperator{\STAB}{STAB}
\DeclareMathOperator{\Down}{Down}
\DeclareMathOperator{\lca}{lca}
\DeclareMathOperator{\ex}{ex}
\DeclareMathOperator{\Span}{span}
\DeclareMathOperator{\T}{T}
\DeclareMathOperator{\rank}{rank}
\DeclareMathOperator{\diag}{diag}
\DeclareMathOperator{\comp}{c}
\DeclareMathOperator{\bernoulli}{Bernoulli}
\newcommand\mycommfont[1]{\footnotesize\ttfamily\textcolor{blue}{#1}}
\SetCommentSty{mycommfont}
\begin{document}
\pagenumbering{arabic}
\title{Homework 1}
\author{Yuchong Pan}
\date{\today}
\newtheorem{theorem}{Theorem}
\newtheorem{lemma}[theorem]{Lemma}
\newtheorem{corollary}[theorem]{Corollary}
\newtheorem{fact}[theorem]{Fact}
\newtheorem{claim}{Claim}
\newtheorem{exercise}{Exercise}
\theoremstyle{definition}
\newtheorem{definition}[theorem]{Definition}
\newtheorem{solution}{Solution}
%\maketitle
%

\begin{framed}
\noindent{\bf 6.842 Randomness and Computation} \hfill \thedate
\begin{center}
\Large{\thetitle}
\end{center}
%\noindent{\em Lecturer: Ronitt Rubinfield} \hfill {\em Scribe: \theauthor}
\noindent{\em Yuchong Pan} \hfill {\em MIT ID: 911346847}
\end{framed}

\begin{enumerate}
  \item \noindent\emph{Collaborators and sources:} none.

  \bigskip

  \begin{proof}
    We construct an approximation scheme $\mathcal B$ for $f$ as follows: On input $(x, \varepsilon, \delta)$, run $\mathcal A(x, \varepsilon)$ independently for $k := \lceil 12\log (1/\delta) \rceil$ times with outputs $y_1, \ldots, y_k$, and output a median of $y_1, \ldots, y_k$.
    
    Let $t_{\mathcal A}(x, \varepsilon)$ be the running time of $\mathcal A$ on input $(x, \varepsilon)$. Then $\mathcal B$ runs in $O(kt_{\mathcal A}(x, \varepsilon)) = O(\log(1/\delta) t_{\mathcal A}(x, \varepsilon))$. Since $\mathcal A$ runs in time polynomial in $1/\varepsilon$ and $|x|$, then $\mathcal B$ runs in time polynomial in $1/\varepsilon$, $|x|$ and $\log(1/\delta)$.

    By the definition of medians, if more than half of $y_1, \ldots, y_k$ fall in $[f(x)/(1 + \varepsilon), f(x)(1 + \varepsilon)]$, then $\mathcal B(x, \varepsilon, \delta) \in [f(x)/(1 + \varepsilon), f(x)(1 + \varepsilon)]$. Let $X_1, \ldots, X_k \in \{ 0, 1 \}$ be random variables so that $X_i = 1$ with probability $p := \PP[\mathcal A(x, \varepsilon) \not \in [f(x)/(1 + \varepsilon), f(x)(1 + \varepsilon)]] \leq 1 - 3/4 = 1/4$. Then $\sum_{i = 1}^k \EE[X_i] = kp \leq k/4$. Therefore,
    \begin{align*}
      &\quad\; \PP\left[\mathcal B(x, \varepsilon, \delta) \not \in \left[\frac{f(x)}{1 + \varepsilon}, f(x)(1 + \varepsilon)\right]\right] \\
      &\leq \PP\left[\text{at least half of $y_1, \ldots, y_k$ do not fall in $\left[\frac{f(x)}{1 + \varepsilon}, f(x)(1 + \varepsilon)\right]$}\right] \\
      &= \PP\left[\sum_{i = 1}^k X_i \geq \frac{k}{2}\right] \\
      &= \PP\left[\sum_{i = 1}^k X_i \geq (1 + 1) \cdot \frac{k}{4}\right] \\
      &\leq e^{-\frac{k/4}{3}} && (\text{Chernoff bound}) \\
      &= e^{-\frac{\left\lceil 12\log \frac{1}{\delta} \right\rceil}{12}} \leq e^{-\frac{12\log \frac{1}{\delta}}{12}} = \delta.
    \end{align*}
    Therefore,
    $$ \PP\left[\mathcal B(x, \varepsilon, \delta) \in \left[\frac{f(x)}{1 + \varepsilon}, f(x)(1 + \varepsilon)\right]\right] = 1 - \PP\left[\mathcal B(x, \varepsilon, \delta) \not \in \left[\frac{f(x)}{1 + \varepsilon}, f(x)(1 + \varepsilon)\right]\right] \geq 1 - \delta. $$
    This completes the proof.
  \end{proof}
\end{enumerate}

\end{document}
