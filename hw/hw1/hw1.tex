
\documentclass[letterpaper, reqno,11pt]{article}
\usepackage[margin=1.0in]{geometry}
\usepackage{color,latexsym,amsmath,amssymb}
\usepackage{fancyhdr}
\usepackage{amsthm}
\usepackage[linesnumbered,lined,boxed,commentsnumbered,noend,noline]{algorithm2e}
\usepackage{dsfont}
\usepackage{graphicx}
\usepackage{hyperref}
\usepackage{bbm}
\usepackage{enumitem}
\usepackage[numbers]{natbib}
\usepackage{framed}
\usepackage{titling}
\usepackage{subcaption}
\usepackage[dvipsnames]{xcolor}
\usepackage{tikz}

\tikzset{invclip/.style={clip,insert path={{[reset cm]
  (-16383.99999pt,-16383.99999pt) rectangle (16383.99999pt,16383.99999pt)}}}}

\allowdisplaybreaks

\newcommand{\RR}{\mathbb{R}}
\newcommand{\CC}{\mathbb{C}}
\newcommand{\ZZ}{\mathbb{Z}}
\newcommand{\QQ}{\mathbb{Q}}
\newcommand{\NN}{\mathbb{N}}
\newcommand{\FF}{\mathbb{F}}
\newcommand{\PP}{\mathbb{P}}
\newcommand{\EE}{\mathbb{E}}
\newcommand{\inci}{\mathds{1}}
\DeclareMathOperator{\conv}{conv}
\DeclareMathOperator{\charcone}{char.cone}
\DeclareMathOperator{\STAB}{STAB}
\DeclareMathOperator{\Down}{Down}
\DeclareMathOperator{\lca}{lca}
\DeclareMathOperator{\ex}{ex}
\DeclareMathOperator{\Span}{span}
\DeclareMathOperator{\T}{T}
\DeclareMathOperator{\rank}{rank}
\DeclareMathOperator{\diag}{diag}
\DeclareMathOperator{\comp}{c}
\DeclareMathOperator{\bernoulli}{Bernoulli}
\DeclareMathOperator{\Mod}{mod}
\newcommand\mycommfont[1]{\footnotesize\ttfamily\textcolor{blue}{#1}}
\SetCommentSty{mycommfont}
\begin{document}
\pagenumbering{arabic}
\title{Homework 1}
\author{Yuchong Pan}
\date{\today}
\newtheorem{theorem}{Theorem}
\newtheorem{lemma}[theorem]{Lemma}
\newtheorem{corollary}[theorem]{Corollary}
\newtheorem{fact}[theorem]{Fact}
\newtheorem{claim}{Claim}
\newtheorem{exercise}{Exercise}
\theoremstyle{definition}
\newtheorem{definition}[theorem]{Definition}
\newtheorem{solution}{Solution}
%\maketitle
%

\begin{framed}
\noindent{\bf 6.842 Randomness and Computation} \hfill \thedate
\begin{center}
\Large{\thetitle}
\end{center}
%\noindent{\em Lecturer: Ronitt Rubinfield} \hfill {\em Scribe: \theauthor}
\noindent{\em Yuchong Pan} \hfill {\em MIT ID: 911346847}
\end{framed}

\begin{enumerate}
  \item \noindent\emph{Collaborators and sources:} none.

  \bigskip

  \begin{proof}
    We construct an approximation scheme $\mathcal B$ for $f$ as follows: On input $(x, \varepsilon, \delta)$, run $\mathcal A(x, \varepsilon)$ independently for $k := \lceil 12\log (1/\delta) \rceil$ times with outputs $y_1, \ldots, y_k$, and output a median of $y_1, \ldots, y_k$.
    
    Let $t_{\mathcal A}(x, \varepsilon)$ be the running time of $\mathcal A$ on input $(x, \varepsilon)$. Then $\mathcal B$ runs in $O(kt_{\mathcal A}(x, \varepsilon)) = O(\log(1/\delta) t_{\mathcal A}(x, \varepsilon))$. Since $\mathcal A$ runs in time polynomial in $1/\varepsilon$ and $|x|$, then $\mathcal B$ runs in time polynomial in $1/\varepsilon$, $|x|$ and $\log(1/\delta)$.

    By the definition of medians, if more than half of $y_1, \ldots, y_k$ fall in $[f(x)/(1 + \varepsilon), f(x)(1 + \varepsilon)]$, then $\mathcal B(x, \varepsilon, \delta) \in [f(x)/(1 + \varepsilon), f(x)(1 + \varepsilon)]$. Let $X_1, \ldots, X_k \in \{ 0, 1 \}$ be random variables so that $X_i = 1$ with probability $p := \PP[\mathcal A(x, \varepsilon) \not \in [f(x)/(1 + \varepsilon), f(x)(1 + \varepsilon)]] \leq 1 - 3/4 = 1/4$. Then $\sum_{i = 1}^k \EE[X_i] = kp \leq k/4$. Therefore,
    \begin{align*}
      &\quad\; \PP\left[\mathcal B(x, \varepsilon, \delta) \not \in \left[\frac{f(x)}{1 + \varepsilon}, f(x)(1 + \varepsilon)\right]\right] \\
      &\leq \PP\left[\text{at least half of $y_1, \ldots, y_k$ do not fall in $\left[\frac{f(x)}{1 + \varepsilon}, f(x)(1 + \varepsilon)\right]$}\right] \\
      &= \PP\left[\sum_{i = 1}^k X_i \geq \frac{k}{2}\right] \\
      &= \PP\left[\sum_{i = 1}^k X_i \geq (1 + 1) \cdot \frac{k}{4}\right] \\
      &\leq e^{-\frac{k/4}{3}} && (\text{Chernoff bound}) \\
      &= e^{-\frac{\left\lceil 12\log \frac{1}{\delta} \right\rceil}{12}} \leq e^{-\frac{12\log \frac{1}{\delta}}{12}} = \delta.
    \end{align*}
    Therefore,
    $$ \PP\left[\mathcal B(x, \varepsilon, \delta) \in \left[\frac{f(x)}{1 + \varepsilon}, f(x)(1 + \varepsilon)\right]\right] = 1 - \PP\left[\mathcal B(x, \varepsilon, \delta) \not \in \left[\frac{f(x)}{1 + \varepsilon}, f(x)(1 + \varepsilon)\right]\right] \geq 1 - \delta. $$
    This completes the proof.
  \end{proof}

  \clearpage

  \item \noindent\emph{Collaborators and sources:} none.

  \bigskip

  \begin{proof}
    Suppose $\binom{m}{t} 2^{1 - \binom{t}{2}} < 1$. To prove $R(t) > m$, it suffices to show that there exists a $2$-edge-coloring of $K_m$ such that for all $S \subset V(K_m)$ of size $t$, $E(K_m[S])$ is not monochromatic. We randomly color the edges of $K_m$ red or blue, independently and equiprobably. For each $S \subset V(K_m)$ of size $t$, there are exactly $2^{\binom{t}{2}}$ two-colorings of $E(K_m[S])$, amongst which two are monochromatic colorings (all red and all blue), so
    $$ \PP\left[\text{$E\left(K_m[S]\right)$ is monochromatic}\right] = \frac{2}{2^{\binom{t}{2}}} = 2^{1 - \binom{t}{2}}. $$
    By the union bound,
    \begin{align*}
      &\quad \; \PP\left[\exists S \subset V\left(K_m\right), |S| = t, \text{$E\left(K_m[S]\right)$ is monochromatic}\right] \\
      &\leq \sum_{\substack{S \subset V\left(K_m\right) \\ |S| = t}} \PP\left[\text{$E\left(K_m[S]\right)$ is monochromatic}\right] \\
      &= \binom{m}{t} 2^{1 - \binom{t}{2}} \\
      &< 1.
    \end{align*}
    Therefore,
    \begin{align*}
      &\quad\; \PP\left[\forall S \subset V\left(K_m\right) \text{of size $t$}, \text{$E\left(K_m[S]\right)$ is not monochromatic}\right] \\
      &= 1 - \PP\left[\exists S \subset V\left(K_m\right), |S| = t, \text{$E\left(K_m[S]\right)$ is monochromatic}\right] \\
      &> 1 - 1 = 0.
    \end{align*}
    This proves that there exists a $2$-edge-coloring of $K_m$ such that for all $S \subset V(K_m)$ of size $t$, $E(K_m[S])$ is not monochromatic. The proof is complete.
  \end{proof}

  \clearpage

  \item \noindent\emph{Collaborators and sources:} none.

  \bigskip
  
  \begin{proof}
    Suppose that a Boolean function $f$ can be computed by a randomized polynomial-sized circuit family $\mathcal C = \{ C_n \}_{n \in \NN}$. Let $n \in \NN$. Let $m \in \NN$ be the number of random input bits. Define a $2^n \times 2^m$ matrix $M$ such that
    \begin{itemize}[itemsep=0pt]
      \item each row represents a possible combination of inputs $x_1, \ldots, x_n \in \{ 0, 1 \}$;
      \item each column represents a possible combination of random input bits $r_1, \ldots, r_m \in \{ 0, 1 \}$;
      \item the entry at the row representing $x_1, \ldots, x_n$ and at the column representing $r_1, \ldots, r_m$ equals the value of $C_n$ on inputs $x_1, \ldots, x_n$ with random input bits $r_1, \ldots, r_m$.
    \end{itemize}
    By the definition of polynomial-sized circuit families, each row has at least half of the entries equal to $1$, so the total number of $1$-entries is at least $2^n \cdot 2^m/2 = 2^{n + m - 1}$. Therefore, there exists a column, representing $r_1^*, \ldots, r_m^*$, with at least half of the entries equal to $1$; otherwise every column has fewer than half of the entries equal to $1$, so the total number of $1$-entries is less than $2^m \cdot 2^n/2 = 2^{n + m - 1}$, a contradiction.

    Construct a deterministic circuit $D_n^{(1)}$ by hard-wiring random input bits $r_1 = r_1^*, \ldots, r_m = r_m^*$. Remove the column representing $r_1^*, \ldots, r_m^*$ and each row representing $x_1, \ldots, x_n$ such that the corresponding entry equals $1$, resulting in a new matrix $M'$. Note that this removes at least half of the rows. We claim that each row of $M'$ has at least half of the entries equal to $1$; to see this, we note that the number of columns is decreased by $1$, and that the number of $1$-entries in each remaining row remains the same. Therefore, we apply the same argument and recurse, until every remaining row is all-zero, obtaining circuits $D_n^{(1)}, \ldots, D_n^{(k)}$. Finally, construct a deterministic circuit $D_n$ by taking the ``or'' of $D_n^{(1)}, \ldots, D_n^{(k)}$.

    Since we remove at least half of the rows in each iteration, then there are at most $O(\log 2^n) = O(n)$ iterations, i.e., $k = O(n)$. Since $C_n$ is polynomial-sized, then the size of $C_n$ is upper bounded by $p(n)$. For each $i \in [k]$, since $D_n^{(i)}$ is obtained by hard-wiring random input bits in $C_n$, then the size of $D_n^{(i)}$ is upper bounded by $p(n)$. Finally, since $D_n$ is obtained by taking the ``or'' of $D_n^{(1)}, \ldots, D_n^{(k)}$, then the size of $D$ is upper bounded by $kp(n) + k = O(np(n))$, so $D_n$ is polynomial-sized.

    Let $\mathcal D = \{ D_n \}_{n \in \NN}$. We show that $\mathcal D$ computes $f$. Let $x_1, \ldots, x_n \in \{ 0, 1 \}$. If $f(x_1, \ldots, x_n) = 0$, then the row of the original matrix $M$ representing $x_1, \ldots, x_n$ is all-zero and hence never removed, so none of $D_n^{(1)}, D_n^{(2)}, \ldots$ outputs $0$ on inputs $x_1, \ldots, x_n$, which implies that $D_n$ outputs $0$. If $f(x_1, \ldots, x_n) = 1$, then the row of the original matrix $M$ representing $x_1, \ldots, x_n$ has at least half of the entries equal to $1$, and is hence removed in some iteration, say the $i^\text{th}$ iteration, so $D_n^{(i)}$ outputs $1$ on inputs $x_1, \ldots, x_n$, which implies that $D_n$ outputs $1$. This shows that $D_n$ outputs $f(x_1, \ldots, x_n)$ for all combinations of inputs $x_1, \ldots, x_n \in \{ 0, 1 \}$. Therefore, $\mathcal D$ is a deterministic polynomial-sized circuit family which computes $f$, completing the proof.
  \end{proof}

  \clearpage

  \item \noindent\emph{Collaborators and sources:} none.

  \bigskip

  \begin{proof}
    Suppose $e(\Delta \delta + 1)(1 - 1/k)^\delta < 1$. WLOG, we assume that each vertex has out-degree $\delta$ by removing an outgoing edge from $v$ whenever a vertex $v$ has out-degree greater than $\delta$; this does not increase the in-degree of any vertex, so the maximum in-degree of the resulting graph is at most $\Delta$.

    Let $f : V \to \{ 0, \ldots, k - 1 \}$ be a random coloring of $V$ obtained by choosing for each $v \in V$, $f(v) \in \{ 0, \ldots, k - 1 \}$ independently according to a uniform distribution. For each $v \in V$, let $A_v$ be the event that there does not exist $u \in V$ such that $(v, u) \in E$ and $f(u) = f(v) + 1 \;\Mod{k}$. For each $v \in V$, we denote by $N^+(v)$ the set of out-neighbors of $v$, and denote by $N^-(v)$ the set of in-neighbors of $v$. Then for each $v \in V$,
    \begin{align*}
      \PP\left[A_v\right] &= \PP\left[f(u) \neq f(v) + 1 \;\Mod{k} \;\;\forall u \in N^+(v)\right] \\
      &= \sum_{i = 0}^{k - 1} \PP[f(v) = i] \PP\left[f(u) \neq f(v) + 1 \;\Mod{k} \;\;\forall u \in N^+(v) \;\middle|\; f(v) = i\right] \\
      &= \sum_{i = 0}^{k - 1} \PP[f(v) = i] \PP\left[f(u) \neq i + 1 \;\Mod{k} \;\;\forall u \in N^+(v)\right] && (\text{independence}) \\
      &= \sum_{i = 0}^{k - 1} \PP[f(v) = i] \prod_{u \in N^+(v)} \PP\left[f(u) \neq i + 1 \;\Mod{k}\right] && (\text{independence}) \\
      &= k \cdot \frac{1}{k} \cdot \left(1 - \frac{1}{k}\right)^\delta = \left(1 - \frac{1}{k}\right)^\delta.
    \end{align*}

    Let $v \in V$. To simplify notations, let $N_*^+(v) = N^+(v) \cup \{ v \}$ and $\overline{N_*^+(v)} = V \setminus N_*^+(v)$. Let $I(v) = \{ u \in V : u \not \in N_*^+(v), N^+(u) \cap N^+(v) = \emptyset \} \subset \overline{N_*^+(v)}$. We show that $A_v$ is independent of all events $A_u$ with $u \in I(v)$. Let $v_1, \ldots, v_\ell \in I(v)$. By total independence, for all $\{ a_w : w \in V \} \in \{ 0, \ldots, k - 1 \}^V$,
    \begin{align*}
      \PP\left[f(w) = a_w \;\forall w \in V \right] &= \prod_{w \in V} \PP\left[f(w) = a_w\right] = \prod_{w \in N_*^+(v)} \PP\left[f(w) = a_w\right] \prod_{w \in \overline{N_*^+(v)}} \PP\left[f(w) = a_w\right] \\
      &= \PP\left[f(w) = a_w \;\forall w \in N_*^+(v)\right] \PP\left[f(w) = a_w \;\forall w \in \overline{N_*^+(v)}\right].
    \end{align*}
    For $i \in [\ell]$, since $v_i \in I(v)$, then $A_{v_i}$ depends upon the values of $f$ on $\overline{N_*^+(v)}$ only. Moreover, $A_v$ depends upon the values of $f$ on $N_*^+(v)$ only. Hence, for all $\{ a_w : w \in V \} \in \{ 0, \ldots, k - 1 \}^V$,
    \begin{align*}
      &\quad\; \PP\left[A_v \cap \bigcap_{i = 1}^\ell A_{v_i} \;\middle|\; f(w) = a_w \;\forall w \in V\right] \\
      &= \left(\prod_{u \in N^+(v)} \mathds 1[a_u \neq a_v + 1 \;\Mod{k}]\right) \left(\prod_{i = 1}^\ell \prod_{u \in N^+\left(v_i\right)} \mathds 1\left[a_u \neq a_{v_i} + 1 \;\Mod{k}\right]\right) \\
      &= \PP\left[A_v \;\middle|\; f(w) = a_w \;\forall w \in V\right] \PP\left[\bigcap_{i = 1}^\ell A_{v_i} \;\middle|\; f(w) = a_w \;\forall w \in V\right] \\
      &= \PP\left[A_v \;\middle|\; f(w) = a_w \;\forall w \in N_*^+(v)\right] \PP\left[\bigcap_{i = 1}^\ell A_{v_i} \;\middle|\; f(w) = a_w \;\forall w \in \overline{N_*^+(v)}\right].
    \end{align*}
    Therefore,
    \begin{align*}
      &\quad\; \PP\left[A_v \cap \bigcap_{i = 1}^\ell A_{v_i}\right] \\
      &= \sum_{\{ a_w \}_{w \in V} \in \{ 0, \ldots, k - 1 \}^V} \PP\left[f(w) = a_w \;\forall w \in V\right] \PP\left[A_v \cap \bigcap_{i = 1}^\ell A_{v_i} \;\middle|\; f(w) = a_w \;\forall w \in V\right] \\
      &= \sum_{\{ a_w \}_{w \in V} \in \{ 0, \ldots, k - 1 \}^V} \PP\left[f(w) = a_w \;\forall w \in N_*^+(v)\right] \PP\left[f(w) = a_w \;\forall w \in \overline{N_*^+(v)}\right] \\
      &\qquad \qquad \qquad \qquad \quad\;\; \PP\left[A_v \;\middle|\; f(w) = a_w \;\forall w \in N_*^+(v)\right] \PP\left[\bigcap_{i = 1}^\ell A_{v_i} \;\middle|\; f(w) = a_w \;\forall w \in \overline{N_*^+(v)}\right] \\
      &= \left(\sum_{\substack{ a_w \in \{ 0, \ldots, k - 1 \} \\ \forall w \in N_*^+(v) }} \PP\left[f(w) = a_w \;\forall w \in N_*^+(v)\right] \PP\left[A_v \;\middle|\; f(w) = a_w \;\forall w \in N_*^+(v)\right]\right) \\
      &\quad\; \left(\sum_{\substack{ a_w \in \{ 0, \ldots, k - 1 \} \\ \forall w \in \overline{N_*^+(v)} }} \PP\left[f(w) = a_w \;\forall w \in \overline{N_*^+(v)}\right] \PP\left[\bigcap_{i = 1}^\ell A_{v_i} \;\middle|\; f(w) = a_w \;\forall w \in \overline{N_*^+(v)}\right]\right) \\
      &= \PP\left[A_v\right] \PP\left[\bigcap_{i = 1}^\ell A_{v_i}\right].
    \end{align*}
    This proves that $A_v$ is independent of all events $A_u$ with $u \in I(v)$. Since $D$ is simple, then the maximum degree of the dependency digraph of $\{ A_v : v \in V \}$ is at most
    \begin{align*}
      \max_{v \in V} |V \setminus (I(v) \cup \{ v \})| &= \left|\left\{ u \in V : u \in N^+(v) \;\vee\; \left(u \neq v \;\wedge\; N^+(u) \cap N^+(v) \neq \emptyset\right) \right\}\right| \\
      &= \left|N^+(v) \cup \bigcup_{u \in N^+(v)} \left(N^-(u) \setminus \{ v \}\right)\right| \\
      &\leq \left|N^+(v)\right| + \sum_{u \in N^+(v)} \left(\left|N^-(u)\right| - 1\right) \\
      &\leq \delta + \delta(\Delta - 1) = \Delta \delta.
    \end{align*}

    Since $e(\Delta \delta + 1) (1 - 1/k)^\delta < 1$, then by the Lov\'{a}sz local lemma,
    $$ \PP[\forall v \in V, \exists u \in N^+(v) \text{ s.t.\ } f(u) = f(v) + 1 \;\Mod{k}] = \PP\left[\bigcap_{v \in V} \overline{A_v}\right] > 0. $$
    This implies that there exists a coloring $f : V \to \{ 0, \ldots, k - 1 \}$ such that for all $v \in V$, there exists $u \in N^+(v)$ such that $f(u) = f(v) + 1 \;\Mod{k}$.
    
    \begin{algorithm}[h]
      $\ell \leftarrow 1$ \\
      pick $v_1 \in V$ arbitrarily \\
      \While{true}{
        pick $u \in N^+(v)$ such that $f(u) = f(v) + 1 \;\Mod{k}$ \\
        \If{$\exists i \in [\ell]$ s.t.\ $v_i = u$}{
          \Return{$(v_i, \ldots, v_\ell, u)$}
        }
      }
      \caption{An algorithm which finds a simple directed cycle whose length is a multiple of $k$, given a digraph $D$ with $e(\Delta \delta + 1)(1 - 1/k)^\delta < 1$ and a coloring $f : V \to \{ 0, \ldots, k - 1 \}$ such that for all $v \in V$, there exists $u \in N^+(v)$ such that $f(u) = f(v) + 1 \;\Mod{k}$, where the out-degree of every vertex in $D$ is $\delta$, and the maximum in-degree of $D$ is $\Delta$.}
      \label{alg:dicycle}
    \end{algorithm}

    Consider Algorithm \ref{alg:dicycle}. Since the number of vertices in $D$ is finite, the algorithm eventually visits a vertex which has been visited before, so Algorithm \ref{alg:dicycle} terminates. We show that the returned walk $(v_i, \ldots, v_\ell, u)$ is a simple directed cycle whose length is a multiple of $k$. By definition, since $u$ is the first vertex on this list that has occurred before, then $(v_i, \ldots, v_\ell, u)$ is a simple directed cycle. Since $f(v_j) = f(v_{j - 1}) + 1 \;\Mod{k}$ for each $j \in \{ i + 1, \ldots, \ell \}$, then an inductive argument implies that
    $$ f\left(v_i\right) = f(u) = f\left(v_\ell\right) + 1 \;\Mod{k} = f\left(v_i\right) + \ell - i + 1 \;\Mod{k}. $$
    Therefore, the length of the simple directed cyle $(v_i, \ldots, v_\ell, u)$, which equals $\ell - i + 1$, is a multiple of $k$. This completes the proof.
  \end{proof}
\end{enumerate}

\end{document}
