
\documentclass[letterpaper, reqno,11pt]{article}
\usepackage[margin=1.0in]{geometry}
\usepackage{color,latexsym,amsmath,amssymb}
\usepackage{fancyhdr}
\usepackage{amsthm}
\usepackage[linesnumbered,lined,boxed,commentsnumbered,noend,noline]{algorithm2e}
\usepackage{dsfont}
\usepackage{graphicx}
\usepackage{hyperref}
\usepackage{bbm}
\usepackage{enumitem}
\usepackage[numbers]{natbib}
\usepackage{framed}
\usepackage{titling}
\usepackage{subcaption}
\usepackage[dvipsnames]{xcolor}
\usepackage{tikz}

\tikzset{invclip/.style={clip,insert path={{[reset cm]
  (-16383.99999pt,-16383.99999pt) rectangle (16383.99999pt,16383.99999pt)}}}}

\allowdisplaybreaks

\newcommand{\RR}{\mathbb{R}}
\newcommand{\CC}{\mathbb{C}}
\newcommand{\ZZ}{\mathbb{Z}}
\newcommand{\QQ}{\mathbb{Q}}
\newcommand{\NN}{\mathbb{N}}
\newcommand{\FF}{\mathbb{F}}
\newcommand{\PP}{\mathop{{}\mathbb{P}}}
\newcommand{\EE}{\mathop{{}\mathbb{E}}}
\newcommand{\inci}{\mathds{1}}
\DeclareMathOperator{\conv}{conv}
\DeclareMathOperator{\charcone}{char.cone}
\DeclareMathOperator{\STAB}{STAB}
\DeclareMathOperator{\Down}{Down}
\DeclareMathOperator{\lca}{lca}
\DeclareMathOperator{\ex}{ex}
\DeclareMathOperator{\Span}{span}
\DeclareMathOperator{\T}{T}
\DeclareMathOperator{\rank}{rank}
\DeclareMathOperator{\diag}{diag}
\DeclareMathOperator{\comp}{c}
\DeclareMathOperator{\bernoulli}{Bernoulli}
\DeclareMathOperator{\Mod}{mod}
\DeclareMathOperator{\sign}{sign}
\DeclareMathOperator{\LowestOne}{\textsc{LowestOne}}
\SetKwFor{RepTimes}{repeat}{times}{end}
\newcommand\mycommfont[1]{\footnotesize\ttfamily\textcolor{blue}{#1}}
\SetCommentSty{mycommfont}
\begin{document}
\pagenumbering{arabic}
\title{Homework 2}
\author{Yuchong Pan}
\date{\today}
\newtheorem{theorem}{Theorem}
\newtheorem{lemma}[theorem]{Lemma}
\newtheorem{corollary}[theorem]{Corollary}
\newtheorem{fact}[theorem]{Fact}
\newtheorem{proposition}[theorem]{Proposition}
\newtheorem{claim}{Claim}
\newtheorem{exercise}{Exercise}
\theoremstyle{definition}
\newtheorem{definition}[theorem]{Definition}
\newtheorem{solution}{Solution}
%\maketitle
%

\begin{framed}
\noindent{\bf 6.842 Randomness and Computation} \hfill \thedate
\begin{center}
\Large{\thetitle}
\end{center}
%\noindent{\em Lecturer: Ronitt Rubinfield} \hfill {\em Scribe: \theauthor}
\noindent{\em Yuchong Pan} \hfill {\em MIT ID: 911346847}
\end{framed}

\begin{enumerate}
  \item \noindent\emph{Collaborators and sources:} Guanghao Ye.
  
  \clearpage

  \item \begin{enumerate}
    \item \noindent\emph{Collaborators and sources:} Guanghao Ye.
    
    \begin{proof}
      Let $\{ x, y \} \subset A$ be such that $x \neq y$. Then for any pairwise independent hash function $h \in H$,
      $$ (h(x), h(y)) \in_U T^2. $$
      Therefore,
      $$ \PP_{h \in_U H}[h(x) = h(y)] = \sum_{z \in T} \PP_{h \in_U H}[(h(x), h(y)) = (z, z)] = \sum_{z \in T} \frac{1}{\left|T^2\right|} = |T| \cdot \frac{1}{|T|^2} = \frac{1}{|T|} = \frac{1}{t}. $$
      It follows that
      \begin{align*}
        \EE_{h \in_U H}[\text{\# colliding pairs for $h$}] &= \EE_{h \in_U H}\left[\sum_{\substack{\{ x, y \} \subset A \\ x \neq y}} \mathds 1_{\text{$\{ x, y \}$ is a colliding pair for $h$}}\right] \\
        &= \sum_{\substack{\{ x, y \} \subset A \\ x \neq y}} \EE_{h \in_U H}\left[\mathds 1_{\text{$\{ x, y \}$ is a colliding pair for $h$}}\right] \\
        &= \sum_{\substack{\{ x, y \} \subset A \\ x \neq y}} \PP_{h \in_U H}\left[\text{$\{ x, y \}$ is a colliding pair for $h$}\right] \\
        &= \sum_{\substack{\{ x, y \} \subset A \\ x \neq y}} \PP_{h \in_U H}[h(x) = h(y)] \\
        &= |\{ \{ x, y \} \subset A : x \neq y \}| \cdot \frac{1}{t} \\
        &= \binom{|A|}{2} \cdot \frac{1}{t} \\
        &= \binom{n}{2} \cdot \frac{1}{t}.
      \end{align*}
      This completes the proof.
    \end{proof}
  \end{enumerate}
\end{enumerate}

\end{document}
