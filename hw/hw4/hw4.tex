
\documentclass[letterpaper, reqno,11pt]{article}
\usepackage[margin=1.0in]{geometry}
\usepackage{color,latexsym,amsmath,amssymb}
\usepackage{fancyhdr}
\usepackage{amsthm}
\usepackage[linesnumbered,lined,boxed,commentsnumbered,noend,noline]{algorithm2e}
\usepackage{dsfont}
\usepackage{graphicx}
\usepackage{hyperref}
\usepackage{bbm}
\usepackage{enumitem}
\usepackage[numbers]{natbib}
\usepackage{framed}
\usepackage{titling}
\usepackage{subcaption}
\usepackage[dvipsnames]{xcolor}
\usepackage{tikz}

\tikzset{invclip/.style={clip,insert path={{[reset cm]
  (-16383.99999pt,-16383.99999pt) rectangle (16383.99999pt,16383.99999pt)}}}}

\allowdisplaybreaks

\newcommand{\RR}{\mathbb{R}}
\newcommand{\CC}{\mathbb{C}}
\newcommand{\ZZ}{\mathbb{Z}}
\newcommand{\QQ}{\mathbb{Q}}
\newcommand{\NN}{\mathbb{N}}
\newcommand{\FF}{\mathbb{F}}
\newcommand{\PP}{\mathop{{}\mathbb{P}}}
\newcommand{\EE}{\mathop{{}\mathbb{E}}}
\newcommand{\inci}{\mathds{1}}
\DeclareMathOperator{\conv}{conv}
\DeclareMathOperator{\charcone}{char.cone}
\DeclareMathOperator{\STAB}{STAB}
\DeclareMathOperator{\Down}{Down}
\DeclareMathOperator{\lca}{lca}
\DeclareMathOperator{\ex}{ex}
\DeclareMathOperator{\Span}{span}
\DeclareMathOperator{\T}{T}
\DeclareMathOperator{\rank}{rank}
\DeclareMathOperator{\diag}{diag}
\DeclareMathOperator{\comp}{c}
\DeclareMathOperator{\bernoulli}{Bernoulli}
\DeclareMathOperator{\Mod}{mod}
\DeclareMathOperator{\sign}{sign}
\DeclareMathOperator{\LowestOne}{\textsc{LowestOne}}
\DeclareMathOperator{\dist}{dist}
\SetKwFor{RepTimes}{repeat}{times}{end}
\newcommand\mycommfont[1]{\footnotesize\ttfamily\textcolor{blue}{#1}}
\SetCommentSty{mycommfont}
\begin{document}
\pagenumbering{arabic}
\title{Homework 4}
\author{Yuchong Pan}
\date{\today}
\newtheorem{theorem}{Theorem}
\newtheorem{lemma}[theorem]{Lemma}
\newtheorem{corollary}[theorem]{Corollary}
\newtheorem{fact}[theorem]{Fact}
\newtheorem{proposition}[theorem]{Proposition}
\newtheorem{claim}{Claim}
\newtheorem{exercise}{Exercise}
\theoremstyle{definition}
\newtheorem{definition}[theorem]{Definition}
\newtheorem{solution}{Solution}
%\maketitle
%

\begin{framed}
\noindent{\bf 6.842 Randomness and Computation} \hfill \thedate
\begin{center}
\Large{\thetitle}
\end{center}
%\noindent{\em Lecturer: Ronitt Rubinfield} \hfill {\em Scribe: \theauthor}
\noindent{\em Yuchong Pan} \hfill {\em MIT ID: 911346847}
\end{framed}

\begin{enumerate}
  \item \noindent\emph{Collaborators and sources:} Guanghao Ye, Zixuan Xu.
  
  \begin{proof}
    
  \end{proof}

  \clearpage

  \item \begin{enumerate}
    \item \noindent\emph{Collaborators and sources:} none.
    
    \begin{proof}
      Note that $\mathds 1_\text{test accepts} = (1 + f(x)f(y)f(z))/2$. By the Fourier transform of $f$ and by linearity of expecation,
      \begin{align*}
        \EE[f(x) f(y) f(z)] &= \EE\left[\left(\sum_{S \subset [n]} \hat{f}(S) \chi_S(x)\right) \left(\sum_{T \subset [n]} \hat{f}(T) \chi_T(y)\right) \left(\sum_{U \subset [n]} \hat{f}(U) \chi_U(z)\right)\right] \\
        &= \sum_{S, T, U \subset [n]} \hat{f}(S) \hat{f}(T) \hat{f}(U) \EE\left[\chi_S(x) \chi_T(y) \chi_U(x \circ y \circ w)\right].
      \end{align*}
      Let $S, T, U \subset [n]$. For all $i \in [n]$, since $x_i, y_i \in \{ \pm 1 \}$, then $x_i^2 = y_i^2 = 1$. Hence,
      \begin{align*}
        \chi_S(x) \chi_T(y) \chi_U(x \circ y \circ w) &= \left(\prod_{i \in S} x_i\right) \left(\prod_{i \in T} y_i\right) \left(\prod_{i \in U} x_i y_i w_i\right) \\
        &= \left(\prod_{i \in S \cap U} x_i^2\right) \left(\prod_{i \in T \cap U} y_i^2\right) \left(\prod_{i \in S \triangle U} x_i\right) \left(\prod_{i \in T \triangle U} y_i\right) \left(\prod_{i \in U} w_i\right) \\
        &= \chi_{S \triangle U}(x) \chi_{T \triangle U}(y) \chi_U(w).
      \end{align*}
      If $S = T = U$, since $w_1, \ldots, w_n$ are all chosen independently and since $\EE[w_i] = (-1) \cdot \delta + 1 \cdot (1 - \delta) = 1 - 2\delta$ for all $i \in [m]$, then
      \begin{align*}
        \EE\left[\chi_{S \triangle U}(x) \chi_{T \triangle U}(y) \chi_U(w)\right] &= \EE\left[\prod_{i \in S} w_i\right] = \prod_{i \in S} \EE\left[w_i\right] = (1 - 2\delta)^{|S|}.
      \end{align*}
      Now, suppose that either $S \neq U$ or $T \neq U$. WLOG assume that $S \neq U$. Then $S \triangle U \neq \emptyset$. Let $j \in S \triangle U$. For $x \in \{ \pm 1 \}^n$, let $x^{\oplus j}$ be the vector obtained by flipping the $j^\text{th}$ bit in $x$. Then we can partition $\{ \pm 1 \}^n$ into (unordered) pairs $(x, x^{\oplus j})$. Therefore,
      \begin{align*}
        \EE\left[\chi_{S \triangle U}(x)\right] &= \frac{1}{2^n} \sum_{x \in \{ \pm 1 \}^n} \chi_{S \triangle U}(x) = \frac{1}{2^n} \sum_{\text{pairs $\left(x, x^{\oplus j}\right)$}} \left(\chi_{S \triangle U}(x) + \chi_{S \triangle U}\left(x^{\oplus j}\right)\right) \\
        &= \frac{1}{2^n} \sum_{\text{pairs $\left(x, x^{\oplus j}\right)$}} \left(x_j \prod_{i \in (S \triangle U) \setminus \{ j \}} x_i + \left(-x_j\right) \prod_{i \in (S \triangle U) \setminus \{ j \}} x_i\right) = 0.
      \end{align*}
      Since $x$, $y$ and $w$ are chosen independently, then for all $S, T, U \subset [n]$ such that either $S \neq U$ or $T \neq U$,
      $$ \EE\left[\chi_{S \triangle U}(x) \chi_{T \triangle U}(y) \chi_U(w)\right] = \EE\left[\chi_{S \triangle U}(x)\right] \EE\left[\chi_{T \triangle U}(y)\right] \EE\left[\chi_U(w)\right] = 0. $$
      Therefore,
      \begin{align*}
        \PP[\text{test accepts}] &= \EE\left[\mathds 1_\text{test accepts}\right] = \EE\left[\frac{1 + f(x)f(y)f(z)}{2}\right] = \frac{1}{2} + \frac{1}{2} \EE[f(x)f(y)f(z)] \\
        &= \frac{1}{2} + \frac{1}{2} \sum_{S, T, U \subset [n]} \hat{f}(S) \hat{f}(T) \hat{f}(U) \EE\left[\chi_{S \triangle U}(x) \chi_{T \triangle U}(y) \chi_U(w)\right] \\
        &= \frac{1}{2} + \frac{1}{2} \sum_{S \subset [n]} (1 - 2\delta)^{|S|} \hat{f}(S)^3.
      \end{align*}
      This completes the proof.
    \end{proof}

    \clearpage

    \item \noindent\emph{Collaborators and sources:} none.
    
    \begin{proof}
      Let $f : \{ \pm 1 \}^n \to \{ \pm 1 \}$ be a dictator function. Then $f = \chi_{\{ j \}}$ for some $j \in [n]$. Therefore, $\hat{f}(\{ j \}) = 1$ and $\hat{f}(S) = 0$ for all $S \subset [n]$ with $S \neq \{ j \}$. By part (a),
      \begin{align*}
        \PP[\text{test accepts}] &= \frac{1}{2} + \frac{1}{2} \sum_{S \subset [n]} (1 - 2\delta)^{|S|} \hat{f}(S)^3 \\
        &= \frac{1}{2} + \frac{1}{2} \left((1 - 2\delta)^{|\{ j \}|} \hat{f}(\{ j \})^3 + \sum_{\substack{S \subset [n] \\ S \neq \{ j \}}} (1 - 2\delta)^{|S|} \hat{f}(S)^3\right) \\
        &= \frac{1}{2} + \frac{1}{2} \left((1 - 2\delta)^1 \cdot 1^3 + \sum_{\substack{S \subset [n] \\ S \neq \{ j \}}} (1 - 2\delta)^{|S|} \cdot 0^3\right) \\
        &= \frac{1}{2} + \frac{1}{2} (1 - 2\delta) = 1 - \delta.
      \end{align*}
      This completes the proof.
    \end{proof}

    \clearpage

    \item \noindent\emph{Collaborators and sources:} none.
    
    \begin{proof}
      Let $f : \{ \pm 1 \}^n \to \{ \pm 1 \}$ be such that $f$ passes with probability at least $1 - \varepsilon$. By part (a),
      $$ 1 - \varepsilon \leq \PP[\text{test accepts}] = \frac{1}{2} + \frac{1}{2} \sum_{S \subset [n]} (1 - 2\delta)^{|S|} \hat{f}(S)^3. $$
      Rearranging the above inequality and applying Parseval's identity yield
      \begin{align*}
        1 - 2\varepsilon &\leq \sum_{S \subset [n]} (1 - 2\delta)^{|S|} \hat{f}(S)^3 \leq \left(\max_{S \subset [n]} (1 - 2\delta)^{|S|} \hat{f}(S)\right) \sum_{S \subset [n]} \hat{f}(S)^2 \\
        &= \left(\max_{S \subset [n]} (1 - 2\delta)^{|S|} \hat{f}(S)\right) \cdot 1 = \max_{S \subset [n]} (1 - 2\delta)^{|S|} \hat{f}(S).
      \end{align*}
      Hence, there exists $S \subset [n]$ such that $(1 - 2\delta)^{|S|} \hat{f}(S) \geq 1 - 2\varepsilon$. Set $\delta = \varepsilon$ in the test. Then $(1 - 2\varepsilon)^{|S|} \hat{f}(S) \geq 1 - 2\varepsilon$. If $|S| \geq 2$, then $0 < (1 - 2\varepsilon)^{|S|} < 1 - 2\varepsilon$, so $(1 - 2\varepsilon)^{|S|} \hat{f}(S) < 1 - 2\varepsilon$ since $\hat{f}(S) = 1 - 2\dist(f, \chi_S) \in [-1, 1]$, a contradiction. Therefore, one of the following two cases holds:
      \begin{enumerate}[label=(\roman*), itemsep=0pt]
        \item $|S| = 1$ and $\hat{f}(S) = 1$;
        \item $|S| = 0$ and $\hat{f}(S) \geq 1 - 2\varepsilon$.
      \end{enumerate}
      This completes the proof.
    \end{proof}
  \end{enumerate}
\end{enumerate}

\end{document}
