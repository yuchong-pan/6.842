
\documentclass[letterpaper, reqno,11pt]{article}
\usepackage[margin=1.0in]{geometry}
\usepackage{color,latexsym,amsmath,amssymb}
\usepackage{fancyhdr}
\usepackage{amsthm}
\usepackage[linesnumbered,lined,boxed,commentsnumbered,noend,noline]{algorithm2e}
\usepackage{dsfont}
\usepackage{graphicx}
\usepackage{hyperref}
\usepackage{bbm}
\usepackage{enumitem}
\usepackage[numbers]{natbib}
\usepackage{framed}
\usepackage{titling}
\usepackage{subcaption}
\usepackage[dvipsnames]{xcolor}
\usepackage{tikz}

\tikzset{invclip/.style={clip,insert path={{[reset cm]
  (-16383.99999pt,-16383.99999pt) rectangle (16383.99999pt,16383.99999pt)}}}}

\allowdisplaybreaks
\DontPrintSemicolon
\SetKw{Break}{break}

\newcommand{\RR}{\mathbb{R}}
\newcommand{\CC}{\mathbb{C}}
\newcommand{\ZZ}{\mathbb{Z}}
\newcommand{\QQ}{\mathbb{Q}}
\newcommand{\NN}{\mathbb{N}}
\newcommand{\FF}{\mathbb{F}}
\newcommand{\PP}{\mathop{{}\mathbb{P}}}
\newcommand{\EE}{\mathop{{}\mathbb{E}}}
\newcommand{\inci}{\mathds{1}}
\DeclareMathOperator{\conv}{conv}
\DeclareMathOperator{\charcone}{char.cone}
\DeclareMathOperator{\STAB}{STAB}
\DeclareMathOperator{\Down}{Down}
\DeclareMathOperator{\lca}{lca}
\DeclareMathOperator{\ex}{ex}
\DeclareMathOperator{\Span}{span}
\DeclareMathOperator{\T}{T}
\DeclareMathOperator{\rank}{rank}
\DeclareMathOperator{\diag}{diag}
\DeclareMathOperator{\comp}{c}
\DeclareMathOperator{\bernoulli}{Bernoulli}
\DeclareMathOperator{\Mod}{mod}
\DeclareMathOperator{\sign}{sign}
\DeclareMathOperator{\LowestOne}{\textsc{LowestOne}}
\DeclareMathOperator{\dist}{dist}
\DeclareMathOperator{\poly}{poly}
\DeclareMathOperator{\argmin}{arg\,min}
\DeclareMathOperator{\error}{error}
\DeclareMathOperator{\NS}{\mathit{NS}}
\DeclareMathOperator{\Inf}{\mathit{Inf}}
\DeclareMathOperator{\Unif}{\mathsf{Unif}}
\SetKwFor{RepTimes}{repeat}{times}{end}
\newcommand\mycommfont[1]{\footnotesize\ttfamily\textcolor{blue}{#1}}
\SetCommentSty{mycommfont}
\begin{document}
\pagenumbering{arabic}
\title{Homework 5 Problem 4}
\author{Yuchong Pan}
\date{\today}
\newtheorem{theorem}{Theorem}
\newtheorem{lemma}[theorem]{Lemma}
\newtheorem{corollary}[theorem]{Corollary}
\newtheorem{fact}[theorem]{Fact}
\newtheorem{proposition}[theorem]{Proposition}
\newtheorem{claim}{Claim}
\newtheorem{exercise}{Exercise}
\theoremstyle{definition}
\newtheorem{definition}[theorem]{Definition}
\newtheorem{solution}{Solution}
%\maketitle
%

\begin{framed}
\noindent{\bf 6.842 Randomness and Computation} \hfill \thedate
\begin{center}
\Large{\thetitle}
\end{center}
%\noindent{\em Lecturer: Ronitt Rubinfield} \hfill {\em Scribe: \theauthor}
\noindent{\em Yuchong Pan, Guanghao Ye} %\hfill {\em MIT ID: 911346847}
\end{framed}

\begin{enumerate}[label=(\alph*)]
  \item \begin{proof}
    Let $X = (X_1, \ldots, X_n) \in \{ \pm 1 \}^n$ be an $(\varepsilon, k)$-wise independent random vector for some $\varepsilon \in (0, 1)$ and $k \in [n]$. Let $S \subset [n]$ be such that $0 < |S| \leq k$. By straightforward calculations (see, e.g., the proof of Problem 1 part (a) in Homework 2), for all $\ell \in [n]$,
    $$ \PP_{\left(W_1, \ldots, W_\ell\right) \sim \Unif\{ \pm 1 \}^\ell}\left[\prod_{i = 1}^\ell W_\ell = 1\right] = \frac{1}{2}. $$
    Since $X$ is $(\varepsilon, k)$-wise independent and since $0 < |S| \leq k$,
    $$ \left|\PP_X\left[\prod_{i \in S} X_i = 1\right] - \frac{1}{2}\right| = \left|\PP_X\left[\prod_{i \in S} X_i = 1\right] - \PP_{\left(W_1, \ldots, W_\ell\right) \sim \Unif\{ \pm 1 \}^\ell}\left[\prod_{i = 1}^\ell W_\ell = 1\right]\right| \leq \varepsilon. $$
    WLOG, assume that
    $$ \PP_X\left[\prod_{i \in S} X_i = 1\right] = \frac{1 + \varepsilon_0}{2}, $$
    for some $\varepsilon_0 \in [0, 2\varepsilon]$ (the case $\PP_X[\prod_{i \in S} X_i = 1] = (1 - \varepsilon_0)/2$ for some $\varepsilon_0 \in [0, 2\varepsilon]$ is symmetric). Let $\lambda = 1/(1 + \varepsilon_0) \in (0, 1]$. Let $Y = (Y_1, \ldots, Y_n) \in \{ \pm 1 \}^n$ be a random vector defined as follows:
    \begin{enumerate}[label=(\roman*), itemsep=0pt]
      \item With probability $\lambda$, let $Y = X$.
      \item With probability $1 - \lambda$, let $Y$ be uniform over
      \begin{equation} \label{eq:4b-case2}
        \mathcal W := \left\{ \left(x_1, \ldots, x_n\right) \in \{ \pm 1 \}^n : \prod_{i \in S} x_i = -1 \right\}.
      \end{equation}
    \end{enumerate}
    Then
    $$ \PP_Y\left[\prod_{i \in S} Y_i = 1\right] = \lambda \PP_X\left[\prod_{i \in S} X_i = 1\right] + (1 - \lambda) \cdot 0 = \frac{1}{1 + \varepsilon_0} \cdot \frac{1 + \varepsilon_0}{2} = \frac{1}{2}. $$
    For all $\mathcal T \subset \{ \pm 1 \}^n$,
    \begin{align*}
      \PP_X[X \in \mathcal T] - \PP_Y[Y \in \mathcal T] &= \PP_X[X \in \mathcal T] - \left(\lambda \PP_X[X \in \mathcal T] + (1 - \lambda) \PP_{W \sim \Unif \mathcal W}[W \in \mathcal T]\right) \\
      &= (1 - \lambda) \left(\PP_X[X \in \mathcal T] - \PP_{W \sim \Unif \mathcal W}[W \in \mathcal T]\right) \\
      &\leq (1 - \lambda) (1 - 0) = 1 - \frac{1}{1 + \varepsilon_0} = \frac{\varepsilon_0}{1 + \varepsilon_0} \\
      &\leq \varepsilon_0 \leq 2\varepsilon.
    \end{align*}
    Therefore,
    $$ \Delta(X, Y) = \max_{\mathcal T \subset \{ \pm 1 \}^n} \left(\PP_X[X \in \mathcal T] - \PP_Y[Y \in \mathcal T]\right) \leq 2\varepsilon. $$
    This completes the proof.
  \end{proof}

  \clearpage

  \item \begin{proof}
    Let $X \in \{ \pm 1 \}^n$ be an $(\varepsilon, k)$-wise independent random vector for some $\varepsilon \in (0, 1)$ and $k \in [n]$. We give a procedure in Algorithm \ref{alg:4b-k-wise} to obtain a $k$-wise independent random vector $Z \in \{ \pm 1 \}^n$ such that $\Delta(X, Z) \leq 2\varepsilon n^k$, where line \ref{line:4b-single-set} uses part (a). We denote by subscript $i \in [n]$ the $i^\text{th}$ coordinate of a vector.

    \begin{algorithm}
      $Y \leftarrow X$ \\
      \ForEach{$S \subset [n]$ with $0 < |S| \leq k$}{
        construct a random vector $Y' \in \{ \pm 1 \}^n$ with $\PP_{Y'}[\prod_{i \in S} Y_i' = 1] = 1/2$ and $\Delta(Y, Y') \leq 2\varepsilon$ \label{line:4b-single-set} \\
        $Y \leftarrow Y'$
      }
      $Z \leftarrow Y$ \\
      \Return{$Z$}
      \caption{A procedure that, given an $(\varepsilon, k)$-wise independent random vector $X \in \{ \pm 1 \}^n$ where $\varepsilon \in (0, 1)$ and $k \in [n]$, returns a $k$-wise independent random vector $Z \in \{ \pm 1 \}^n$ such that $\Delta(X, Z) \leq 2\varepsilon n^k$.}
      \label{alg:4b-k-wise}
    \end{algorithm}

    Note that $\Delta(Y, Y') \leq 2\varepsilon$ during each iteration. Let $(S_1, \ldots, S_\ell)$ be an enumeration of subsets $S \subset [n]$ with $0 < |S| \leq k$. By the triangle inequality,
    \begin{align}
      \Delta(X, Z) &\leq \Delta\left(X, S_1\right) + \sum_{i = 2}^\ell \Delta\left(S_{i - 1}, S_i\right) \nonumber \\
      &= 2\varepsilon |\{ S \subset [n] : 0 < |S| \leq k \}| \nonumber \\
      &\leq 2\varepsilon \left|[n]^k\right| = 2\varepsilon n^k. \label{eq:4b-count}
    \end{align}
    Note that \eqref{eq:4b-count} follows from the fact that each $k$-tuple $(i_1, \ldots, i_k) \in [n]^k$ corresponds to a subset $S = \{ i_1, \ldots, i_k \} \subset [n]$ with $0 < |S| \leq k$.

    Now, we prove that $Z$ is $k$-wise independent. Since $\PP_{Y'}[\prod_{i \in S} Y_i' = 1] = 1/2$ in the iteration for each subset $S \subset [n]$ with $0 < |S| \leq k$, then it suffices to show that the iteration for a subset $S \subset [n]$ with $0 < |S| \leq k$ does not increase $|\PP_{Y'}[\prod_{i \in T} Y_i'] - 1/2|$ for all $T \subset [n]$ with $0 < |T| \leq k$ and $S \neq T$. To see this, we first note that
    $$ \PP_{W \sim \Unif \mathcal W}\left[\sum_{i \in T} W_i = 1\right] = \frac{1}{2}, $$
    where $\mathcal W$ is defined as in \eqref{eq:4b-case2}, by straightforward calculations (see, e.g., the proof of Problem 1 part (a) in Homework 2). Then
    \begin{align*}
      \left|\PP_{Y'}\left[\prod_{i \in T} Y_i' = 1\right] - \frac{1}{2}\right| &= \left|\left(\lambda\PP_Y\left[\prod_{i \in T} Y_i = 1\right] + (1 - \lambda) \cdot \frac{1}{2}\right) - \frac{1}{2}\right| \\
      &= \left|\lambda\left(\PP_Y\left[\prod_{i \in T} Y_i = 1\right] - \frac{1}{2}\right)\right| = \lambda \left|\PP_Y\left[\prod_{i \in T} Y_i = 1\right] - \frac{1}{2}\right|.
    \end{align*}
    Since $\lambda \in (0, 1]$, then this shows that the iteration for a subset $S \subset [n]$ with $0 < |S| \leq k$ does not increase $|\PP_{Y'}[\prod_{i \in T} Y_i'] - 1/2|$ for all $T \subset [n]$ with $0 < |T| \leq k$ and $S \neq T$. Hence, at the end of the procedure, $\PP_Z[\prod_{i \in S} Z_i] = 1/2$ for all $S \subset [n]$ with $0 < |S| \leq k$. This shows that $Z$ is $k$-wise independent, completing the proof.
  \end{proof}
\end{enumerate}

\end{document}
