
\documentclass[letterpaper, reqno,11pt]{article}
\usepackage[margin=1.0in]{geometry}
\usepackage{color,latexsym,amsmath,amssymb}
\usepackage{fancyhdr}
\usepackage{amsthm}
\usepackage[linesnumbered,lined,boxed,commentsnumbered,noend,noline]{algorithm2e}
\usepackage{dsfont}
\usepackage{graphicx}
\usepackage{hyperref}
\usepackage{bbm}
\usepackage{enumitem}
\usepackage[numbers]{natbib}
\usepackage{framed}
\usepackage{titling}
\usepackage{subcaption}
\usepackage[dvipsnames]{xcolor}
\usepackage{tikz}

\tikzset{invclip/.style={clip,insert path={{[reset cm]
  (-16383.99999pt,-16383.99999pt) rectangle (16383.99999pt,16383.99999pt)}}}}

\allowdisplaybreaks

\newcommand{\RR}{\mathbb{R}}
\newcommand{\CC}{\mathbb{C}}
\newcommand{\ZZ}{\mathbb{Z}}
\newcommand{\QQ}{\mathbb{Q}}
\newcommand{\NN}{\mathbb{N}}
\newcommand{\FF}{\mathbb{F}}
\newcommand{\PP}{\mathop{{}\mathbb{P}}}
\newcommand{\EE}{\mathop{{}\mathbb{E}}}
\newcommand{\inci}{\mathds{1}}
\DeclareMathOperator{\conv}{conv}
\DeclareMathOperator{\charcone}{char.cone}
\DeclareMathOperator{\STAB}{STAB}
\DeclareMathOperator{\Down}{Down}
\DeclareMathOperator{\lca}{lca}
\DeclareMathOperator{\ex}{ex}
\DeclareMathOperator{\Span}{span}
\DeclareMathOperator{\T}{T}
\DeclareMathOperator{\rank}{rank}
\DeclareMathOperator{\diag}{diag}
\DeclareMathOperator{\comp}{c}
\DeclareMathOperator{\bernoulli}{Bernoulli}
\DeclareMathOperator{\Mod}{mod}
\newcommand\mycommfont[1]{\footnotesize\ttfamily\textcolor{blue}{#1}}
\SetCommentSty{mycommfont}
\begin{document}
\pagenumbering{arabic}
\title{Homework 2}
\author{Yuchong Pan}
\date{\today}
\newtheorem{theorem}{Theorem}
\newtheorem{lemma}[theorem]{Lemma}
\newtheorem{corollary}[theorem]{Corollary}
\newtheorem{fact}[theorem]{Fact}
\newtheorem{proposition}[theorem]{Proposition}
\newtheorem{claim}{Claim}
\newtheorem{exercise}{Exercise}
\theoremstyle{definition}
\newtheorem{definition}[theorem]{Definition}
\newtheorem{solution}{Solution}
%\maketitle
%

\begin{framed}
\noindent{\bf 6.842 Randomness and Computation} \hfill \thedate
\begin{center}
\Large{\thetitle}
\end{center}
%\noindent{\em Lecturer: Ronitt Rubinfield} \hfill {\em Scribe: \theauthor}
\noindent{\em Yuchong Pan} \hfill {\em MIT ID: 911346847}
\end{framed}

\begin{enumerate}
  \item \begin{enumerate}
    \item \noindent\emph{Collaborators and sources:} none.

    \bigskip

    \begin{proof}
      Recall that the $n = 2^\ell - 1$ pairwise independent random bits are generated by $C_S = \prod_{i \in S} b_i$ for all $S \subset [\ell]$ with $S \neq \emptyset$, from $\ell$ truly random bits $b_1, \ldots, b_\ell \in \{ -1, 1 \}$. First, we show that $\PP[C_S = 1] = \PP[C_S = -1] = 1/2$ for all $S \subset [\ell]$ with $S \neq \emptyset$. Let $b \in \{ -1, 1 \}$. Let $S \subset [\ell]$ be such that $S \neq \emptyset$. Then
      \begin{align*}
        \PP\left[C_S = 1\right] &= \frac{1}{2^{|S|}} \sum_{i = 1}^{\left\lceil \frac{|S|}{2} \right\rceil} \binom{|S|}{2i - 1} \\
        &= \left\{
          \begin{array}{ll}
            \frac{1}{2^{|S|}} \sum_{i = 1}^{|S|/2} \left(\binom{|S| - 1}{2i - 2} + \binom{|S| - 1}{2i - 1}\right), & \text{if $|S|$ is even}, \\
            \frac{1}{2^{|S|}} \left(\sum_{i = 1}^{(|S| - 1)/2} \left(\binom{|S| - 1}{2i - 2} + \binom{|S| - 1}{2i - 1}\right) + \binom{|S|}{|S|}\right), & \text{if $|S|$ is odd},
          \end{array}
        \right. \\
        &= \left\{
          \begin{array}{ll}
            \frac{1}{2^{|S|}} \sum_{i = 0}^{|S| - 1} \binom{|S| - 1}{i}, & \text{if $|S|$ is even}, \\
            \frac{1}{2^{|S|}} \left(\sum_{i = 0}^{|S| - 2} \binom{|S| - 1}{i} + \binom{|S| - 1}{|S| - 1}\right), & \text{if $|S|$ is odd},
          \end{array}
        \right. \\
        &= \frac{1}{2^{|S|}} \sum_{i = 0}^{|S| - 1} \binom{|S| - 1}{i} = \frac{2^{|S| - 1}}{2^{|S|}} = \frac{1}{2}.
      \end{align*}
      Hence, $\PP[C_S = -1] = 1 - \PP[C_S = 1] = 1 - 1/2 = 1/2$.

      Now, let $S, S' \subset [\ell]$ be such that $S \neq S'$, $S \neq \emptyset$ and $S' \neq \emptyset$. Let $b, b' \in \{ -1, 1 \}$. Then
      \begin{align}
        \PP\left[C_S = b, C_{S'} = b'\right] &= \sum_{\beta \in \{ -1, 1 \}} \PP\left[C_{S \cap S'} = \beta\right] \PP\left[C_S = b, C_{S'} = b' \;\middle|\; C_{S \cap S'} = \beta\right] \nonumber \\
        &= \sum_{\beta \in \{ -1, 1 \}} \PP\left[C_{S \cap S'} = \beta\right] \PP\left[C_{S \setminus S'} = b \beta, C_{S' \setminus S} = b' \beta\right] \nonumber \\
        &= \sum_{\beta \in \{ -1, 1 \}} \PP\left[C_{S \cap S'} = \beta\right] \PP\left[C_{S \setminus S'} = b \beta\right] \PP\left[C_{S' \setminus S} = b' \beta\right] \label{eq:1a-indep} \\
        &= \sum_{\beta \in \{ -1, 1 \}} \frac{1}{2} \cdot \frac{1}{2} \cdot \frac{1}{2} = 2 \cdot \frac{1}{8} = \frac{1}{4} = \frac{1}{2} \cdot \frac{1}{2} = \PP\left[C_S = b\right] \PP\left[C_S = b'\right]. \nonumber
      \end{align}
      Note that \eqref{eq:1a-indep} follows from the fact that $S \setminus S'$ and $S' \setminus S$ are disjoint and thus that $C_{S \setminus S'}$ and $C_{S' \setminus S}$ are independent. This completes the proof that the $n = 2^{\ell} - 1$ random bits $C_S$ for $S \subset [\ell]$ with $S \neq \emptyset$ are pairwise independent.
    \end{proof}

    \clearpage

    \item \noindent\emph{Collaborators and sources:} none.

    \bigskip

    For each $i \in [s], j \in [n]$, we denote by $s_{i, j}$ the $(i, j)$-entry of $\mathsf{S}$. For each $j \in [n]$, we denote by $\mathbf s_j$ the $j^\text{th}$ column of $\mathsf{S}$. The condition of pairwise independence says that for all $j, j' \in [n]$ with $j \neq j'$ and for all $b, b' \in \{ -1, 1 \}$,
    \begin{equation} \label{eq:1b-pw-indep}
      \PP_{i \in [s]} \left[s_{i, j} = b, s_{i, j'} = b'\right] = \PP_{i \in [s]}\left[\mathbf x_j^{(i)} = b, \mathbf x_{j'}^{(i)} = b'\right] = \frac{1}{4}.
    \end{equation}

    We show that $S$ contains at least $n$ vectors.

    \begin{proof}
      WLOG, assume that $n \geq 2$ and that $s \geq 1$. First, we show that $\mathbf s_j \cdot \mathbf s_{j'} = 0$ for all $j, j' \in [n]$ with $j \neq j'$. Let $j, j' \in [n]$ be such that $j \neq j'$. Since $S$ is a pairwise independent space, then \eqref{eq:1b-pw-indep} implies that for all $b, b' \in \{ -1, 1 \}$,
      $$ \left|\left\{ i \in [s] : s_{i, j} = b, s_{i, j'} = b' \right\}\right| = \frac{s}{4}. $$
      Therefore,
      \begin{align*}
        \mathbf s_j \cdot \mathbf s_{j'} &= \sum_{i = 1}^s s_{i, j} s_{i, j'} = \left|\left\{ i \in [s] : s_{i, j} = s_{i, j'} \right\}\right| - \left|\left\{ i \in [s] : s_{i, j} \neq s_{i, j'} \right\}\right| \\
        &= \left(\left|\left\{ i \in [s] : s_{i, j} = s_{i, j'} = 1 \right\}\right| + \left|\left\{ i \in [s] : s_{i, j} = s_{i, j'} = -1 \right\}\right|\right) - \\
        &\quad\, \left(\left|\left\{ i \in [s] : s_{i, j} = 1, s_{i, j'} = -1 \right\}\right| + \left|\left\{ i \in [s] : s_{i, j} = -1, s_{i, j'} = 1 \right\}\right|\right) \\
        &= \left(\frac{s}{4} + \frac{s}{4}\right) - \left(\frac{s}{4} + \frac{s}{4}\right) = 0.
      \end{align*}

      Second, we show that $\mathbf s_1, \ldots, \mathbf s_n$ are linearly independent. Suppose for the purpose of contradiction that there exist $\alpha_1, \ldots, \alpha_n \in \RR$ that are not all zeros such that
      $$ \sum_{j = 1}^n \alpha_j \mathbf s_j = \mathbf 0. $$
      Let $j' \in [n]$. Since $|\{ i \in [s] : s_{i, j} = 1, s_{i, j'} = 1 \}| = s/4 > 0$ for all $j \in [n] \setminus \{ j' \}$, then $\mathbf s_{j'} \neq \mathbf 0$ and hence $\|\mathbf s_{j'}\|^2 > 0$. Therefore,
      \begin{align*}
        0 &= \mathbf 0 \cdot \mathbf s_{j'} = \left(\sum_{j = 1}^n \alpha_j \mathbf s_j\right) \cdot \mathbf s_{j'} = \sum_{j = 1}^n \alpha_j \left(\mathbf s_j \cdot \mathbf s_{j'}\right) = \sum_{\substack{j = 1 \\ j \neq j'}}^n \alpha_j \left(\mathbf s_j \cdot \mathbf s_{j'}\right) + \alpha_{j'} \left(\mathbf s_{j'} \cdot \mathbf s_{j'}\right) \\
        &= \sum_{\substack{j = 1 \\ j \neq j'}}^n \alpha_j \cdot 0 + \alpha_{j'} \left\| \mathbf s_{j'} \right\|^2 = \alpha_{j'} \left\| \mathbf s_{j'} \right\|^2.
      \end{align*}
      This implies that $\alpha_{j'} = 0/\| \mathbf s_{j'} \|^2 = 0$ for all $j' \in [n]$, a contradiction. Hence, $\mathbf s_1, \ldots, \mathbf s_n$ are linearly independent. It follows that
      $$ s \geq \rank \mathsf{S} = n. $$
      This completes the proof.
    \end{proof}

    \clearpage

    \item \noindent\emph{Collaborators and sources:} none.

    \bigskip

    \begin{proof}
      Note that any algorithm which generates $n$ pairwise independent random bits samples a vector $\mathbf x$ from a pairwise independent space $S = \{ \mathbf x^{(1)}, \ldots, \mathbf x^{(s)} \}$ on $n$ variables. By part (b), any pairwise independent space $S$ on $n$ variables has size $|S| \geq n$. Therefore, any algorithm that generates $n$ pairwise independent random bits requires at least $\log n$ truly random bits to sample a vector from a space of size $n$. This implies that the construction is optimal, completing the proof.
    \end{proof}
  \end{enumerate}
\end{enumerate}

\end{document}
