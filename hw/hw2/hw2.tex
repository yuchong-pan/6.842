
\documentclass[letterpaper, reqno,11pt]{article}
\usepackage[margin=1.0in]{geometry}
\usepackage{color,latexsym,amsmath,amssymb}
\usepackage{fancyhdr}
\usepackage{amsthm}
\usepackage[linesnumbered,lined,boxed,commentsnumbered,noend,noline]{algorithm2e}
\usepackage{dsfont}
\usepackage{graphicx}
\usepackage{hyperref}
\usepackage{bbm}
\usepackage{enumitem}
\usepackage[numbers]{natbib}
\usepackage{framed}
\usepackage{titling}
\usepackage{subcaption}
\usepackage[dvipsnames]{xcolor}
\usepackage{tikz}

\tikzset{invclip/.style={clip,insert path={{[reset cm]
  (-16383.99999pt,-16383.99999pt) rectangle (16383.99999pt,16383.99999pt)}}}}

\allowdisplaybreaks

\newcommand{\RR}{\mathbb{R}}
\newcommand{\CC}{\mathbb{C}}
\newcommand{\ZZ}{\mathbb{Z}}
\newcommand{\QQ}{\mathbb{Q}}
\newcommand{\NN}{\mathbb{N}}
\newcommand{\FF}{\mathbb{F}}
\newcommand{\PP}{\mathop{{}\mathbb{P}}}
\newcommand{\EE}{\mathop{{}\mathbb{E}}}
\newcommand{\inci}{\mathds{1}}
\DeclareMathOperator{\conv}{conv}
\DeclareMathOperator{\charcone}{char.cone}
\DeclareMathOperator{\STAB}{STAB}
\DeclareMathOperator{\Down}{Down}
\DeclareMathOperator{\lca}{lca}
\DeclareMathOperator{\ex}{ex}
\DeclareMathOperator{\Span}{span}
\DeclareMathOperator{\T}{T}
\DeclareMathOperator{\rank}{rank}
\DeclareMathOperator{\diag}{diag}
\DeclareMathOperator{\comp}{c}
\DeclareMathOperator{\bernoulli}{Bernoulli}
\DeclareMathOperator{\Mod}{mod}
\DeclareMathOperator{\sign}{sign}
\SetKwFor{RepTimes}{repeat}{times}{end}
\newcommand\mycommfont[1]{\footnotesize\ttfamily\textcolor{blue}{#1}}
\SetCommentSty{mycommfont}
\begin{document}
\pagenumbering{arabic}
\title{Homework 2}
\author{Yuchong Pan}
\date{\today}
\newtheorem{theorem}{Theorem}
\newtheorem{lemma}[theorem]{Lemma}
\newtheorem{corollary}[theorem]{Corollary}
\newtheorem{fact}[theorem]{Fact}
\newtheorem{proposition}[theorem]{Proposition}
\newtheorem{claim}{Claim}
\newtheorem{exercise}{Exercise}
\theoremstyle{definition}
\newtheorem{definition}[theorem]{Definition}
\newtheorem{solution}{Solution}
%\maketitle
%

\begin{framed}
\noindent{\bf 6.842 Randomness and Computation} \hfill \thedate
\begin{center}
\Large{\thetitle}
\end{center}
%\noindent{\em Lecturer: Ronitt Rubinfield} \hfill {\em Scribe: \theauthor}
\noindent{\em Yuchong Pan} \hfill {\em MIT ID: 911346847}
\end{framed}

\begin{enumerate}
  \item \begin{enumerate}
    \item \noindent\emph{Collaborators and sources:} none.

    \bigskip

    \begin{proof}
      Recall that the $n = 2^\ell - 1$ pairwise independent random bits are generated by $C_S = \prod_{i \in S} b_i$ for all $S \subset [\ell]$ with $S \neq \emptyset$, from $\ell$ truly random bits $b_1, \ldots, b_\ell \in \{ -1, 1 \}$. First, we show that $\PP[C_S = 1] = \PP[C_S = -1] = 1/2$ for all $S \subset [\ell]$ with $S \neq \emptyset$. Let $b \in \{ -1, 1 \}$. Let $S \subset [\ell]$ be such that $S \neq \emptyset$. Then
      \begin{align*}
        \PP\left[C_S = 1\right] &= \frac{1}{2^{|S|}} \sum_{i = 1}^{\left\lceil \frac{|S|}{2} \right\rceil} \binom{|S|}{2i - 1} \\
        &= \left\{
          \begin{array}{ll}
            \frac{1}{2^{|S|}} \sum_{i = 1}^{|S|/2} \left(\binom{|S| - 1}{2i - 2} + \binom{|S| - 1}{2i - 1}\right), & \text{if $|S|$ is even}, \\
            \frac{1}{2^{|S|}} \left(\sum_{i = 1}^{(|S| - 1)/2} \left(\binom{|S| - 1}{2i - 2} + \binom{|S| - 1}{2i - 1}\right) + \binom{|S|}{|S|}\right), & \text{if $|S|$ is odd},
          \end{array}
        \right. \\
        &= \left\{
          \begin{array}{ll}
            \frac{1}{2^{|S|}} \sum_{i = 0}^{|S| - 1} \binom{|S| - 1}{i}, & \text{if $|S|$ is even}, \\
            \frac{1}{2^{|S|}} \left(\sum_{i = 0}^{|S| - 2} \binom{|S| - 1}{i} + \binom{|S| - 1}{|S| - 1}\right), & \text{if $|S|$ is odd},
          \end{array}
        \right. \\
        &= \frac{1}{2^{|S|}} \sum_{i = 0}^{|S| - 1} \binom{|S| - 1}{i} = \frac{2^{|S| - 1}}{2^{|S|}} = \frac{1}{2}.
      \end{align*}
      Hence, $\PP[C_S = -1] = 1 - \PP[C_S = 1] = 1 - 1/2 = 1/2$.

      Now, let $S, S' \subset [\ell]$ be such that $S \neq S'$, $S \neq \emptyset$ and $S' \neq \emptyset$. Let $b, b' \in \{ -1, 1 \}$. Then
      \begin{align}
        \PP\left[C_S = b, C_{S'} = b'\right] &= \sum_{\beta \in \{ -1, 1 \}} \PP\left[C_{S \cap S'} = \beta\right] \PP\left[C_S = b, C_{S'} = b' \;\middle|\; C_{S \cap S'} = \beta\right] \nonumber \\
        &= \sum_{\beta \in \{ -1, 1 \}} \PP\left[C_{S \cap S'} = \beta\right] \PP\left[C_{S \setminus S'} = b \beta, C_{S' \setminus S} = b' \beta\right] \nonumber \\
        &= \sum_{\beta \in \{ -1, 1 \}} \PP\left[C_{S \cap S'} = \beta\right] \PP\left[C_{S \setminus S'} = b \beta\right] \PP\left[C_{S' \setminus S} = b' \beta\right] \label{eq:1a-indep} \\
        &= \sum_{\beta \in \{ -1, 1 \}} \frac{1}{2} \cdot \frac{1}{2} \cdot \frac{1}{2} = 2 \cdot \frac{1}{8} = \frac{1}{4} = \frac{1}{2} \cdot \frac{1}{2} = \PP\left[C_S = b\right] \PP\left[C_S = b'\right]. \nonumber
      \end{align}
      Note that \eqref{eq:1a-indep} follows from the fact that $S \setminus S'$ and $S' \setminus S$ are disjoint and thus that $C_{S \setminus S'}$ and $C_{S' \setminus S}$ are independent. This completes the proof that the $n = 2^{\ell} - 1$ random bits $C_S$ for $S \subset [\ell]$ with $S \neq \emptyset$ are pairwise independent.
    \end{proof}

    \clearpage

    \item \noindent\emph{Collaborators and sources:} Guanghao Ye.

    \bigskip

    We show that
    \begin{enumerate}[itemsep=0pt, label=(\roman*)]
      \item a \emph{necessary} condition of $S$ being a pairwise independent space is that the columns of $\mathsf{S}$ are pairwise orthogonal;
      \item a pairwise independent space $S$ contains at least $n$ vectors.
    \end{enumerate}

    \begin{proof}
      WLOG, assume that $n \geq 2$ and that $s \geq 1$. For each $i \in [s], j \in [n]$, we denote by $s_{i, j}$ the $(i, j)$-entry of $\mathsf{S}$. For each $j \in [n]$, we denote by $\mathbf s_j$ the $j^\text{th}$ column of $\mathsf{S}$.
      
      \begin{enumerate}[label=(\roman*)]
        \item Suppose that $S$ is a pairwise independent space. Let $j, j' \in [n]$ be such that $j \neq j'$. Then for all $b, b' \in \{ -1, 1 \}$,
        $$ \PP_{i \in [s]} \left[s_{i, j} = b, s_{i, j'} = b'\right] = \PP_{i \in [s]}\left[\mathbf x_j^{(i)} = b, \mathbf x_{j'}^{(i)} = b'\right] = \frac{1}{4}, $$
        and hence,
        $$ \left|\left\{ i \in [s] : s_{i, j} = b, s_{i, j'} = b' \right\}\right| = \frac{s}{4}. $$
        Therefore,
        \begin{align*}
          \mathbf s_j \cdot \mathbf s_{j'} &= \sum_{i = 1}^s s_{i, j} s_{i, j'} = \left|\left\{ i \in [s] : s_{i, j} = s_{i, j'} \right\}\right| - \left|\left\{ i \in [s] : s_{i, j} \neq s_{i, j'} \right\}\right| \\
          &= \left(\left|\left\{ i \in [s] : s_{i, j} = s_{i, j'} = 1 \right\}\right| + \left|\left\{ i \in [s] : s_{i, j} = s_{i, j'} = -1 \right\}\right|\right) - \\
          &\quad\, \left(\left|\left\{ i \in [s] : s_{i, j} = 1, s_{i, j'} = -1 \right\}\right| + \left|\left\{ i \in [s] : s_{i, j} = -1, s_{i, j'} = 1 \right\}\right|\right) \\
          &= \left(\frac{s}{4} + \frac{s}{4}\right) - \left(\frac{s}{4} + \frac{s}{4}\right) = 0.
        \end{align*}
        \item We show that $\mathbf s_1, \ldots, \mathbf s_n$ are linearly independent. Suppose that $S$ is a pairwise independent space. Suppose for the sake of contradiction that there exist $\alpha_1, \ldots, \alpha_n \in \RR$ that are not all zeros such that
        $$ \sum_{j = 1}^n \alpha_j \mathbf s_j = \mathbf 0. $$
        Let $j' \in [n]$. Since $|\{ i \in [s] : s_{i, j} = 1, s_{i, j'} = 1 \}| = s/4 > 0$ for all $j \in [n] \setminus \{ j' \}$, then $\mathbf s_{j'} \neq \mathbf 0$ and hence $\|\mathbf s_{j'}\|^2 > 0$. Therefore,
        \begin{align*}
          0 &= \mathbf 0 \cdot \mathbf s_{j'} = \left(\sum_{j = 1}^n \alpha_j \mathbf s_j\right) \cdot \mathbf s_{j'} = \sum_{j = 1}^n \alpha_j \left(\mathbf s_j \cdot \mathbf s_{j'}\right) = \sum_{\substack{j = 1 \\ j \neq j'}}^n \alpha_j \left(\mathbf s_j \cdot \mathbf s_{j'}\right) + \alpha_{j'} \left(\mathbf s_{j'} \cdot \mathbf s_{j'}\right) \\
          &= \sum_{\substack{j = 1 \\ j \neq j'}}^n \alpha_j \cdot 0 + \alpha_{j'} \left\| \mathbf s_{j'} \right\|^2 = \alpha_{j'} \left\| \mathbf s_{j'} \right\|^2.
        \end{align*}
        This implies that $\alpha_{j'} = 0/\| \mathbf s_{j'} \|^2 = 0$ for all $j' \in [n]$, a contradiction. Hence, $\mathbf s_1, \ldots, \mathbf s_n$ are linearly independent. It follows that
        $$ s \geq \rank \mathsf{S} = n. $$
        This completes the proof.
      \end{enumerate}
    \end{proof}

    \clearpage

    \item \noindent\emph{Collaborators and sources:} none.

    \bigskip

    \begin{proof}
      Note that any algorithm which generates $n$ pairwise independent random bits samples a vector $\mathbf x$ from a pairwise independent space $S = \{ \mathbf x^{(1)}, \ldots, \mathbf x^{(s)} \}$ on $n$ variables. By part (b), any pairwise independent space $S$ on $n$ variables has size $|S| \geq n$. Therefore, any algorithm that generates $n$ pairwise independent random bits requires at least $\log n$ truly random bits to sample a vector from a space of size $n$. This implies that the construction is optimal, completing the proof.
    \end{proof}
  \end{enumerate}

  \clearpage

  \item \begin{enumerate}
    \item \noindent\emph{Collaborators and sources:} Guanghao Ye.

    \bigskip

    \begin{proof}
      Let $x \in [n]$. Since $w_x$ is chosen from $S$ uniformly at random, then for all $s \in \ZZ$,
      $$ \PP\left[w_x = s\right] = \left\{
        \begin{array}{ll}
          0, & \text{if $s \not \in S$}, \\
          \frac{1}{|S|}, & \text{if $s \in S$},
        \end{array}
      \right\} \leq \frac{1}{|S|}. $$
      Therefore,
      $$ \PP[\alpha(x) = w_x] = \PP\left[w_x = \min_{\substack{i \in [k] \\ x \not \in M_i}} w\left(M_i\right) - \min_{\substack{i \in [k] \\ x \in M_i}} w\left(M_i \setminus \{ x \}\right)\right] \leq \frac{1}{|S|}. $$
      By the union bound,
      $$ \PP\left[\exists x \in [n] \text{ such that } \alpha(x) = w_x\right] \leq \sum_{x = 1}^n \PP\left[\alpha(x) = w_x\right] \leq n \cdot \frac{1}{|S|} = \frac{n}{|S|}. $$
      This completes the proof.
    \end{proof}

    \clearpage

    \item \noindent\emph{Collaborators and sources:} Guanghao Ye.

    \bigskip

    \begin{proof}
      Suppose that there exist two distinct $M_j$ and $M_\ell$ with $j, \ell \in [k]$ that have the same minimum weight (compared to all other $w(M_i)$ with $i \in [k]$). Then there exists $x \in M_j \triangle M_\ell$. WLOG, suppose that $x \not \in M_j$ and $x \in M_\ell$. Since $M_j$ and $M_\ell$ have the same minimum weight, then
      \begin{gather*}
        w\left(M_j\right) = w\left(M_\ell\right), \\
        \min_{i \in [k], x \not \in M_i} w(M_i) = w(M_j), \\
        \min_{i \in [k], x \in M_i} w(M_i) = w(M_\ell).
      \end{gather*}
      Hence,
      \begin{align*}
        \alpha(x) &= \min_{\substack{i \in [k] \\ x \not \in M_i}} w\left(M_i\right) - \min_{\substack{i \in [k] \\ x \in M_i}} w\left(M_i \setminus \{ x \}\right) = \min_{\substack{i \in [k] \\ x \not \in M_i}} w\left(M_i\right) - \min_{\substack{i \in [k] \\ x \in M_i}} \left(w\left(M_i\right) - w_x\right) \\
        &= \min_{\substack{i \in [k] \\ x \not \in M_i}} w\left(M_i\right) - \min_{\substack{i \in [k] \\ x \in M_i}} w\left(M_i\right) + w_x = w\left(M_j\right) - w\left(M_\ell\right) + w_x = w_x.
      \end{align*}
      This implies that
      \begin{align}
        &\quad\; \PP\left[\exists \text{a unique $w(M_i)$ with $i \in [k]$ of minimum weight}\right] \nonumber \\
        &= 1 - \PP\left[\exists \text{distinct $M_j, M_\ell$ with $j, \ell \in [k]$ that have the same minimum weight}\right] \nonumber \\
        &\geq 1 - \PP\left[\exists x \in [n] \text{ such that $\alpha(x) = w_x$}\right] \nonumber \\
        &\geq 1 - \frac{n}{|S|}. \label{eq:2a}
      \end{align}
      Note that \eqref{eq:2a} follows from part (a). This completes the proof.
    \end{proof}
  \end{enumerate}

  \clearpage
  
  \item \begin{enumerate}
    \item \noindent\emph{Collaborators and sources:} Guanghao Ye.

    \bigskip

    \begin{proof}
      Let $\mathcal B$ be the sequential algorithm given in Algorithm \ref{alg:find-pm} for finding a perfect matching in a bipartite graph, given a black box algorithm $\mathcal A$ that checks whether a given bipartite graph contains a perfect matching or not. In other words, for each edge $e \in E$, $\mathcal B$ checks whether the graph $G''$ obtained by removing $e$ and its endpoints from $G$ has a perfect matching; if so, then $\mathcal B$ replaces the current graph with $G''$; otherwise, $\mathcal B$ removes $e$ from the current graph (and keeps its endpoints).

      \begin{algorithm}
        \If{$\mathcal A(G) = 0$}{
          \Return{``$G$ does not have a perfect matching''}
        }
        $M = \emptyset$ \\
        $G' \leftarrow G$ \\
        \ForEach{$e = (u, v) \in E$}{
          $G'' \leftarrow (V(G') \setminus \{ u, v \}, E(G') \setminus \{ e \})$ \\
          \If{$\mathcal A(G'') = 0$}{
            $M \leftarrow M \cup \{ e \}$ \\
            $G' \leftarrow G''$
          }
          \Else{
            $G' \leftarrow (V(G'), E(G') \setminus \{ e \})$
          }
        }
        \Return{$M$}
        \caption{A sequential algorithm for finding a perfect matching in a bipartite graph $G = (V, E)$, given a black box algorithm $\mathcal A$ that checks whether a given bipartite graph contains a perfect matching.}
        \label{alg:find-pm}
      \end{algorithm}

      Since $\mathcal B$ makes $m + 1$ calls to $\mathcal A$, then $\mathcal B$ runs in time $O((m + 1) \cdot T_{\mathcal A}^\textit{seq}(G)) = O(m \cdot T_{\mathcal A}^\textit{seq}(G))$. We show that $\mathcal B$ is correct. If $G$ does not contain a perfect matching, then $\mathcal B$ correctly reports so. Suppose that $G$ contains a perfect matching. If an edge $e \in E$ is in a perfect matching of $G$, then the graph $G''$ obtained by removing $e$ and its endpoints from $G$ contains a perfect matching $M'$ such that $M' \cup \{ e \}$ is a perfect matching of $G$; otherwise, any perfect matching of $G$ still exists if we remove $e$ (and keep its endpoints). This justifies the correctness of $\mathcal B$, completing the proof.
    \end{proof}

    \clearpage

    \item \noindent\emph{Collaborators and sources:} Guanghao Ye.

    \bigskip

    \begin{proof}
      We claim that, with high probability, Algorithm \ref{alg:find-pm-parallel} correctly finds the unique perfect matching in a bipartite graph $G$ that has exactly one perfect matching, given an oracle for determinant computations, such that all calls to the oracle are simultaneous.

      \begin{algorithm}
        $\mathcal T \leftarrow \emptyset$ \\
        pick $n^4$ random points $S = \{ s_1, \ldots, s_{n^4}$ \} \\
        \ForEach{$e = (u, v) \in E$}{
          $E' \leftarrow E \setminus \{ e \}$ \\
          \ForEach{$(u', v') \in E'$}{
            pick $x_{u', v'} \in S$ uniformly at random
          }
          let $A \in \RR^{n \times n}$ such that $\forall u, v \in V$, $A_{u', v'}$ equals $x_{u', v'}$ if $(u', v') \in E'$ and $0$ otherwise \\
          $\mathcal T \leftarrow \mathcal T \cup \{ \text{task to compute $\det A$ and save the result to variable $d_e$} \}$
        }
        run tasks in $\mathcal T$ in parallel, obtaining variables $d_e$ for $e \in E$ \\
        $M \leftarrow \{ e \in E : d_e = 0 \}$ \\
        \Return{$M$}
        \caption{An algorithm for finding the unique perfect matching in a bipartite graph $G = (V, E)$ that has exactly one perfect matching, given an oracle for determinant computations, such that all calls to the oracle are simultaneous.}
        \label{alg:find-pm-parallel}
      \end{algorithm}

      Note that if an edge $e \in E$ is contained in the unique perfect matching of $G$, then the graph $G_e$ obtained by removing $e$ (and keeping its endpoints) from $G$ contains no perfect matching; otherwise, $G_e$ stills contains the unique perfect matching. Recall that $G_e$ contains a perfect matching if and only if its (symbolic) Tutte matrix has a determinant that is a non-zero multivariate polynomial. We denote by $T_e$ the (symbolic) Tutte matrix of $G_e$. This implies that an edge $e \in E$ is contained in the unique perfect matching of $G$ if and only if $\det T_e \equiv 0$. For each $e \in E$ with $\det T_e \equiv 0$, we always have $d_e = 0$ and hence $e \in M$, so $\PP[e \not \in M] = 0$. Therefore,
      \begin{align}
        \PP[\text{Algorithm \ref{alg:find-pm-parallel} is correct}] &= 1 - \PP\left[\text{$\exists e \in M, \det T_e \not\equiv 0$ or $\exists e \in E \setminus M, \det T_e \equiv 0$}\right] \nonumber \\
        &\geq 1 - \sum_{\substack{e \in E \\ \det T_e \not\equiv 0}} \PP\left[e \in M\right] - \sum_{\substack{e \in E \\ \det T_e \equiv 0}} \PP[e \not \in M] \label{eq:3b-union} \\
        &= 1 - \sum_{\substack{e \in E \\ \det T_e \not\equiv 0}} \PP\left[d_e = 0\right] - \sum_{\substack{e \in E \\ \det T_e \equiv 0}} 0 \nonumber \\
        &\geq 1 - \sum_{\substack{e \in E \\ \det T_e \not\equiv 0}} \frac{n}{|S|} \label{eq:3b-schwartz} \\
        &\geq 1 - \sum_{e \in E} \frac{n}{n^4} \nonumber = 1 - m \cdot \frac{1}{n^3} \geq 1 - n^2 \cdot \frac{1}{n^3} = 1 - \frac{1}{n} = 1 - o(1). \nonumber
      \end{align}
      Note that \eqref{eq:3b-union} follows from the union bound, and that \eqref{eq:3b-schwartz} follows from the Schwartz-Zippel-DeMill-Lipton theorem and the fact that the determinant of the Tutte matrix of an $n$-vertex bipartite graph is a multivariate polynomial of total degree at most $n$. This completes the proof.
    \end{proof}

    \clearpage

    \item \noindent\emph{Collaborators and sources:} Guanghao Ye.

    \bigskip

    We show that if $G$ contains only one perfect matching of minimum weight equal to $w^*$ and if the binary representation of $|\det A|$ is $|\det A| = \sum_{i = 0}^k b_i 2^i$, where $b_i \in \{ 0, 1 \}$ for all $i \in \{ 0, \ldots, k \}$, then $w^*$ is the smallest $i \in \{ 0, \ldots, k \}$ such that $b_i = 1$.

    \begin{proof}
      For each $M \subset E$, we denote $w(M) = \sum_{(u, v) \in M} w_{u, v}$. For each permutation $\sigma$ of $[n]$, we deonte by $M_\sigma = \{ (u, \sigma(u)) : u \in L \}$, and we have
      $$ \prod_{u \in L} A_{u, \sigma(u)} = \left\{
        \begin{array}{ll}
          \prod_{u \in L} 2^{w_{u, \sigma(u)}} = 2^{\sum_{u \in L} w_{u, \sigma(u)}} = 2^{w\left(M_\sigma\right)}, & \text{if $M_\sigma$ is a perfect matching}, \\
          0, & \text{otherwise}.
        \end{array}
      \right. $$
      Let $\sigma_1, \ldots, \sigma_\ell$ be permutations of $[n]$ such that $M_{\sigma_i}$ is a perfect matching for each $i \in [\ell]$ and that $w(M_{\sigma_1}) < w(M_{\sigma_2}) \leq \ldots \leq w(M_{\sigma_\ell})$. Then
      \begin{align*}
        \det A &= \sum_{\text{$\sigma$ permutation of $[n]$}} \sign(\sigma) \prod_{u \in L} A_{u, \sigma(u)}, \\
        &= \sum_{\substack{\text{$\sigma$ permutation of $[n]$} \\ \text{$M_\sigma$ perfect matching}}} \sign(\sigma) \prod_{u \in L} A_{u, \sigma(u)} + \sum_{\substack{\text{$\sigma$ permutation of $[n]$} \\ \text{$M_\sigma$ not perfect matching}}} \sign(\sigma) \prod_{u \in L} A_{u, \sigma(u)} \\
        &= \sum_{\substack{\text{$\sigma$ permutation of $[n]$} \\ \text{$M_\sigma$ perfect matching}}} \sign(\sigma) \cdot 2^{w\left(M_\sigma\right)} + \sum_{\substack{\text{$\sigma$ permutation of $[n]$} \\ \text{$M_\sigma$ not perfect matching}}} \sign(\sigma) \cdot 0 \\
        &= \sum_{\substack{\text{$\sigma$ permutation of $[n]$} \\ \text{$M_\sigma$ perfect matching}}} \sign(\sigma) \cdot 2^{w\left(M_\sigma\right)} \\
        &= \sum_{i = 1}^\ell \sign\left(\sigma_i\right) \cdot 2^{w\left(M_{\sigma_i}\right)} \\
        &= 2^{w\left(M_{\sigma_1}\right)} \left(\sign\left(\sigma_1\right) + \sum_{i = 2}^\ell \sign\left(\sigma_i\right) 2^{w\left(M_{\sigma_i}\right) - w\left(M_{\sigma_1}\right)}\right).
      \end{align*}
      Therefore,
      $$ |\det A| = 2^{w\left(M_{\sigma_1}\right)} \left|\sign\left(\sigma_1\right) + \sum_{i = 2}^\ell \sign\left(\sigma_i\right) 2^{w\left(M_{\sigma_i}\right) - w\left(M_{\sigma_1}\right)}\right|. $$
      For each $i \in \{ 2, \ldots, \ell \}$, since $w(M_{\sigma_i}) > w(M_{\sigma_1})$ and since weights are integers, then $2^{w(M_{\sigma_i}) - w(M_{\sigma_1})}$ is even. Since $\sign(\sigma_i) \in \{ 1, -1 \}$ for all $i \in [\ell]$, then
      \begin{equation} \label{eq:3c-odd}
        \left|\sign\left(\sigma_1\right) + \sum_{i = 2}^\ell \sign\left(\sigma_i\right) 2^{w\left(M_{\sigma_i}\right) - w\left(M_{\sigma_1}\right)}\right|
      \end{equation}
      is odd. Suppose that the binary representation of \eqref{eq:3c-odd} is $1 + \sum_{i = 1}^{k'} \beta_i 2^i$, where $\beta_i \in \{ 0, 1 \}$ for all $i \in [k]$. Therefore,
      $$ |\det A| = 2^{w\left(M_{\sigma_1}\right)} \left(1 + \sum_{i = 1}^{k'} \beta_i 2^i\right) = 2^{w\left(M_{\sigma_1}\right)} + \sum_{i = 1}^{k'} \beta_i 2^{w\left(M_{\sigma_1}\right) + i}. $$
      This completes the proof by the uniqueness of binary representations.
    \end{proof}

    \clearpage

    \item \noindent\emph{Collaborators and sources:} Guanghao Ye.

    \bigskip

    The property from part (c) does not necessarily hold if the random weights $w_{u, v}$ for $(u, v) \in E$ result in more than one perfect matching of minimum weight.

    \begin{proof}
      Consider $K_{2, 2}$, i.e., $L = R = [2]$ and $E = L \times R$. Let $w_{u, v} = 1$ for all $(u, v) \in E$. There are two perfect matchings of this graph, namely $\{ (1, 1), (2, 2) \}$ and $\{ (1, 2), (2, 1) \}$, both of weight $2$. Hence, the weights $w_{u, v}$ for $(u, v) \in E$ result in more than one perfect matching of minimum weight. Note that $A_{u, v} = X_{u, v} = 2^{w_{u, v}} = 2^1 = 2$ for all $(u, v) \in L \times R = E$. Therefore,
      $$ \det A = \det \begin{pmatrix}
        2 & 2 \\
        2 & 2
      \end{pmatrix} = 2 \cdot 2 - 2 \cdot 2 = 0. $$
      This shows that the property from part (c) does not hold, completing the proof.
    \end{proof}
  \end{enumerate}
\end{enumerate}

\end{document}
