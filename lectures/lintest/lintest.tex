
\documentclass[letterpaper, reqno,11pt]{article}
\usepackage[margin=1.0in]{geometry}
\usepackage{color,latexsym,amsmath,amssymb}
\usepackage{fancyhdr}
\usepackage{amsthm}
\usepackage[linesnumbered,lined,boxed,commentsnumbered,noend,noline]{algorithm2e}
\usepackage{dsfont}
\usepackage{graphicx}
\usepackage{hyperref}
\usepackage{bbm}
\usepackage[inline]{enumitem}
\usepackage[numbers]{natbib}
\usepackage{framed}
\usepackage{titling}
\usepackage{subcaption}
\usepackage[dvipsnames]{xcolor}
\usepackage{tikz}

\tikzset{invclip/.style={clip,insert path={{[reset cm]
  (-16383.99999pt,-16383.99999pt) rectangle (16383.99999pt,16383.99999pt)}}}}

\allowdisplaybreaks

\newcommand{\RR}{\mathbb{R}}
\newcommand{\CC}{\mathbb{C}}
\newcommand{\ZZ}{\mathbb{Z}}
\newcommand{\QQ}{\mathbb{Q}}
\newcommand{\NN}{\mathbb{N}}
\newcommand{\FF}{\mathbb{F}}
\newcommand{\PP}{\mathop{{}\mathbb{P}}}
\newcommand{\EE}{\mathop{{}\mathbb{E}}}
\newcommand{\LL}{\mathbb{L}}
\newcommand{\TT}{\mathbb{T}}
\newcommand{\GI}{\textrm{GI}}
\newcommand{\coGI}{\overline{\textrm{GI}}}
\DeclareMathOperator{\conv}{conv}
\DeclareMathOperator{\charcone}{char.cone}
\DeclareMathOperator{\STAB}{STAB}
\DeclareMathOperator{\Down}{Down}
\DeclareMathOperator{\lca}{lca}
\DeclareMathOperator{\ex}{ex}
\DeclareMathOperator{\Span}{span}
\DeclareMathOperator{\T}{T}
\DeclareMathOperator{\F}{F}
\DeclareMathOperator{\shP}{\# P}
\DeclareMathOperator{\shSAT}{\# SAT}
\DeclareMathOperator{\shDNF}{\# DNF}
\DeclareMathOperator{\DNF}{DNF}
\DeclareMathOperator{\Poly}{P}
\DeclareMathOperator{\CNF}{CNF}
\DeclareMathOperator{\SAT}{SAT}
\DeclareMathOperator{\BPP}{BPP}
\DeclareMathOperator{\poly}{poly}
\DeclareMathOperator{\RP}{RP}
\DeclareMathOperator{\EXP}{EXP}
\DeclareMathOperator{\DTIME}{DTIME}
\DeclareMathOperator{\NP}{NP}
\DeclareMathOperator{\MCprime}{MC'}
\DeclareMathOperator{\Var}{Var}
\DeclareMathOperator{\IP}{IP}
\DeclareMathOperator{\PSPACE}{PSPACE}
\DeclareMathOperator{\lollipop}{lollipop}
\DeclareMathOperator{\ustconn}{\textsc{UST-Conn}}
\DeclareMathOperator{\RL}{RL}
\newcommand\mycommfont[1]{\ttfamily\textcolor{blue}{#1}}
\SetCommentSty{mycommfont}
\SetKwFor{RepTimes}{repeat}{times}{end}
\begin{document}
\pagenumbering{arabic}
\title{Lectures on Linearity Testing}
\author{Yuchong Pan}
\date{\today}
\newtheorem{theorem}{Theorem}
\newtheorem{lemma}[theorem]{Lemma}
\newtheorem{proposition}[theorem]{Proposition}
\newtheorem{corollary}[theorem]{Corollary}
\newtheorem{fact}[theorem]{Fact}
\newtheorem{problem}[theorem]{Problem}
\newtheorem{claim}{Claim}
\newtheorem{exercise}{Exercise}
\theoremstyle{definition}
\newtheorem{definition}[theorem]{Definition}
%\maketitle
%

\begin{framed}
\noindent{\bf 6.842 Randomness and Computation} \hfill \thedate
\begin{center}
\Large{\thetitle}
\end{center}
\noindent{\em Lecturer: Ronitt Rubinfield} \hfill {\em Scribe: \theauthor}
\end{framed}

\begin{definition}
  Let $G$ and $H$ be finite groups. Let $f : G \to H$. Then $f$ is said to be \emph{linear} (i.e., is a \emph{homomorphism}) if for all $x, y \in G$,
  $$ f(x) +_H f(y) = f\left(x +_G y\right). $$
  For all $\varepsilon > 0$, $f$ is said to be \emph{$\varepsilon$-linear} if there exists a linear function $g : G \to H$ such that $f$ and $g$ agree on at least $1 - \varepsilon$ fraction of inputs in $G$, i.e.,
  $$ \PP_{x \in G}[f(x) = g(x)] \geq 1 - \varepsilon, $$
  or equivalently,
  $$ \frac{|\{ x \in G : f(x) = g(x) \}|}{|G|} \geq 1 - \varepsilon. $$
\end{definition}

Algorithm \ref{alg:natural-lintest} is a natural test for the linearity of a function $f : G \to H$, where $G$ and $H$ are finite groups.

\begin{algorithm}
  \RepTimes{?}{
    pick random $x, y \in G$ \\
    test $f(x) + f(y) = f(x + y)$
  }
  \caption{A natural test for the linearity of a function $f : G \to H$, where $G$ and $H$ are finite groups.}
  \label{alg:natural-lintest}
\end{algorithm}

\end{document}
