
\documentclass[letterpaper, reqno,11pt]{article}
\usepackage[margin=1.0in]{geometry}
\usepackage{color,latexsym,amsmath,amssymb}
\usepackage{fancyhdr}
\usepackage{amsthm}
\usepackage[linesnumbered,lined,boxed,commentsnumbered,noend,noline]{algorithm2e}
\usepackage{dsfont}
\usepackage{graphicx}
\usepackage{hyperref}
\usepackage{bbm}
\usepackage[inline]{enumitem}
\usepackage[numbers]{natbib}
\usepackage{framed}
\usepackage{titling}
\usepackage{subcaption}
\usepackage[dvipsnames]{xcolor}
\usepackage{tikz}

\tikzset{invclip/.style={clip,insert path={{[reset cm]
  (-16383.99999pt,-16383.99999pt) rectangle (16383.99999pt,16383.99999pt)}}}}

\allowdisplaybreaks

\newcommand{\RR}{\mathbb{R}}
\newcommand{\CC}{\mathbb{C}}
\newcommand{\ZZ}{\mathbb{Z}}
\newcommand{\QQ}{\mathbb{Q}}
\newcommand{\NN}{\mathbb{N}}
\newcommand{\FF}{\mathbb{F}}
\newcommand{\PP}{\mathbb{P}}
\newcommand{\EE}{\mathbb{E}}
\newcommand{\LL}{\mathbb{L}}
\newcommand{\TT}{\mathbb{T}}
\DeclareMathOperator{\conv}{conv}
\DeclareMathOperator{\charcone}{char.cone}
\DeclareMathOperator{\STAB}{STAB}
\DeclareMathOperator{\Down}{Down}
\DeclareMathOperator{\lca}{lca}
\DeclareMathOperator{\ex}{ex}
\DeclareMathOperator{\Span}{span}
\DeclareMathOperator{\T}{T}
\DeclareMathOperator{\F}{F}
\DeclareMathOperator{\shP}{\# P}
\DeclareMathOperator{\shSAT}{\# SAT}
\DeclareMathOperator{\shDNF}{\# DNF}
\DeclareMathOperator{\DNF}{DNF}
\DeclareMathOperator{\Poly}{P}
\DeclareMathOperator{\CNF}{CNF}
\DeclareMathOperator{\SAT}{SAT}
\DeclareMathOperator{\BPP}{BPP}
\DeclareMathOperator{\poly}{poly}
\DeclareMathOperator{\RP}{RP}
\DeclareMathOperator{\EXP}{EXP}
\DeclareMathOperator{\DTIME}{DTIME}
\newcommand\mycommfont[1]{\ttfamily\textcolor{blue}{#1}}
\SetCommentSty{mycommfont}
\begin{document}
\pagenumbering{arabic}
\title{Lectures on Derandomization}
\author{Yuchong Pan}
\date{\today}
\newtheorem{theorem}{Theorem}
\newtheorem{lemma}[theorem]{Lemma}
\newtheorem{proposition}[theorem]{Proposition}
\newtheorem{corollary}[theorem]{Corollary}
\newtheorem{fact}[theorem]{Fact}
\newtheorem{claim}{Claim}
\newtheorem{exercise}{Exercise}
\theoremstyle{definition}
\newtheorem{definition}[theorem]{Definition}
%\maketitle
%

\begin{framed}
\noindent{\bf 6.842 Randomness and Computation} \hfill \thedate
\begin{center}
\Large{\thetitle}
\end{center}
\noindent{\em Lecturer: Ronitt Rubinfield} \hfill {\em Scribe: \theauthor}
\end{framed}

\section{Randomized Complexity Class}

\begin{definition}
  A \emph{language} is a subset of $\{ 0, 1 \}^*$.
\end{definition}

\begin{definition}
  $\Poly$ is a complexity class that consists of all languages $L$ with a polynomial time deterministic algorithm $A$.
\end{definition}

\begin{definition}
  $\RP$ is a complexity class that consists of all languages $L$ with a polynomial time probabilistic algorithm $A$ such that
  \begin{align*}
    & \PP[\text{$A$ accepts $x$}] \geq 1/2, & \text{if $x \in L$}, \\
    & \PP[\text{$A$ rejects $x$}] = 1, & \text{if $x \not \in L$},
  \end{align*}
  This is called \emph{$1$-sided error}.
\end{definition}

\begin{definition}
  $\BPP$ is a complexity class that consists of all languages $L$ with a polynomial time probabilistic algorithm $A$ such that
  \begin{align*}
    & \PP[\text{$A$ accepts $x$}] \geq 2/3, & \text{if $x \in L$}, \\
    & \PP[\text{$A$ rejects $x$}] \geq 2/3, & \text{if $x \not \in L$},
  \end{align*}
  This is called \emph{$2$-sided error}.
\end{definition}

\section{Derandomization via Enumeration}

Consider a problem $L$ in $\BPP$. Given a randomized algorithm $A$ that decides $L$ with running time $t(n)$ and $r(n) \leq t(n)$ random bits, we can define a deterministic algorithm in Algorithm \ref{alg:derand-enum} that decides $L$. By the definition of $\BPP$, the majority answer is the correct answer. The running time of Algorithm \ref{alg:derand-enum} is $2^{r(n)} \cdot t(n)$.

\begin{algorithm}
  run $A$ on every possible random string of length $r(n)$ \\
  output the majority answer
  \caption{A deterministic algorithm that derandomizes a randomized algorithm $A$ with running time $t(n)$ and $r(n) \leq t(n)$ random bits.}
  \label{alg:derand-enum}
\end{algorithm}

\begin{definition}
  $\EXP = \bigcup_c \EXP(2^{n^c})$.
\end{definition}

\begin{corollary}
  $\BPP \subseteq \EXP$.
\end{corollary}

\end{document}
